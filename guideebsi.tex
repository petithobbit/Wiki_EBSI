
        %documentlatex pour production du guide de l'étudiant version 1.0
        %%préambule
        \documentclass [12 pt]{article}
        
        %typographie française
        \usepackage [utf8]{inputenc}
        \usepackage [french]{babel} 
        \usepackage [T1]{fontenc} 
       
        
        %mise en page améliorée
        \usepackage [margin=3cm,includefoot]{geometry}
        \usepackage{fancyhdr}
        \pagestyle{fancy}
        \fancyhead[R]{EBSI-Guide de l'étudiant 2016-2017}
        \fancyfoot{}
        \fancyfoot[R]{\thepage\ }
        \renewcommand{\footrulewidth}{1pt}
      
        
        %utilisation d'images
        \usepackage {graphicx} 
        \graphicspath { {/images} }
        
        %préservation des hyperliens (possibilités de renvois internes avec ce package aussi)
        %liens pdf déréférençables avec pdflatex
        \usepackage {hyperref}
        
        %pour test : génération de texte automatique
        \usepackage{blindtext}
        
        %%fin du préambule
        
        \begin{document}
        
        
    
        \newpage
        \section {
        Guide de l'étudiant du 1er cycle 2016-2017
        }
        
        
        
        \textit{
        Guide de l'étudiant du 1er cycle - EBSI - Université de Montréal -
            Confluence
        }
    
        
        %page titre
        \begin{titlepage}
        \begin{center}
        %\includegraphic
        \line(1,0){350}\\
        \huge{\bfseries 
        Guide de l'étudiant du 1er cycle 2016-2017
        }
        \line(1,0){350}
        \vskip10cm
        
        
        \end{center}
        \end{titlepage}
    
         
    
    
        \newpage
        \section {
        Introduction (1er cycle)
        }
        
        
        
        \textit{
        Guides > Introduction (1er cycle) - EBSI - Université de Montréal -
            Confluence
        }
    
        Introduction (1er cycle)
        
            Le personnel de l'École de bibliothéconomie et des sciences de l'information (EBSI)
                souhaite la bienvenue aux étudiants du premier cycle. Tout au long de vos études,
                vous aurez à vous référer à ce guide conçu pour vous donner des informations
                pertinentes. Il contient les principaux renseignements sur :
            
        \begin{itemize}
        
                
        \item le certificat en archivistique;
                
        \item le certificat en gestion de l'information numérique;
                
        \item le cheminement administratif;
                
        \item les politiques, règlements et directives;
                
        \item les personnes ressources.
            
        \end{itemize}
    
            Nous souhaitons que vos études à l'EBSI se déroulent dans un climat chaleureux et
                qu'elles soient couronnées de succès.
            L'École de bibliothéconomie et des sciences de l'information est une école
                professionnelle. Elle est la seule école francophone du domaine en Amérique du Nord.
                Rattachée à l'Université de Montréal depuis 1961, elle a graduellement élargi son
                champ d'action, de la bibliothéconomie à l'archivistique puis à l'ensemble des
                sciences de l'information. L'École fait partie de la Faculté des arts et des
                sciences (FAS) de l'Université de Montréal. Elle entretient d'étroites relations
                avec l'ensemble des écoles canadiennes et américaines en sciences de l'information.
                Elle est également ouverte sur le monde et très présente dans la francophonie.
            L'École de bibliothéconomie et des sciences de l'information offre différents
                programmes de formation :
            
        \begin{itemize}
        
                
        \item Premier cycleCertificat en archivistiqueCertificat en gestion
                            de l'information numérique
                
        \item Deuxième cycleMaîtrise en sciences de l'information
                
        \item Troisième cycleDoctorat en sciences de l'information
            
        \end{itemize}
    
            Précisons que le certificat en archivistique a vu le jour en 1984, le doctorat en
                1997 et le certificat en gestion de l'information numérique en 2001.
            Le directeur de l'École de bibliothéconomie et des
                sciences de l'information de l'Université de Montréal
        
    
    
        \newpage
        \section {
        Place d'un certificat dans un cheminement universitaire (1er cycle)
        }
        
        
        
        \textit{
        Guides > Place d'un certificat dans un cheminement universitaire (1er cycle)
            - EBSI - Université de Montréal - Confluence
        }
    
        Place d'un certificat dans un cheminement universitaire (1er cycle)
        
            Les certificats peuvent s'inscrire au sein d'un 1er cycle universitaire
                complet, c'est-à-dire un baccalauréat, de deux manières :
            
        \begin{itemize}
        
                
        \item par cumul de trois certificats (ou trois mineures), ou,
                
        \item avec un certificat et une majeure.
            
        \end{itemize}
    
            Les certificats peuvent aussi venir compléter des études antérieures ou de
                l'expérience dans des domaines reliés, dans une optique de formation continue ou de
                mise à niveau.
            
                
                    
                
            
             
        
    
    
        \newpage
        \section {
        Présentation de l'EBSI
        }
        
        
        
        \textit{
        Guides > Renseignements généraux > Présentation de l'EBSI - EBSI -
            Université de Montréal - Confluence
        }
    
        Présentation de l'EBSI
        
            
        \textbf{
        Mission
        }
    
            Dans le respect de la mission de l'Université de Montréal, l'EBSI a pour mission de
                :
            
        \begin{itemize}
        
                
        \item Former des professionnels et des chercheurs à la gestion de l'information
                    consignée et des connaissances en vue de maximiser leur transfert et leur
                    utilisation dans la société.
                
        \item Contribuer à l'avancement des connaissances et des pratiques de l'information
                    par ses activités de recherche aux plans national et international.
            
        \end{itemize}
    
            
        \textbf{
        Historique
        }
    
            L'École de bibliothéconomie et des sciences de l'information (EBSI) est une école
                professionnelle de la Faculté des arts et des sciences (FAS) dont les programmes de
                maîtrise en sciences de l'information et de doctorat en sciences de l'information
                relèvent de la Faculté des études supérieures et postdoctorales (FESP).
            
                
                    
                        
                            
                        
                        
                    
                
                
                    The Maîtrise en sciences de l'information
                        program at Université de Montréal is accredited by the American Library
                        Association Committee on Accreditation, with the status of Continued
                            accreditation. The next comprehensive review visit is scheduled
                        for Fall 2021.
                
                Le programme de Maîtrise en sciences de l'information
                    de l'Université de Montréal est agréé par le Committee on Accreditation de
                    l'American Library Association, avec le statut Agrément
                    continu. La prochaine visite d'évaluation complète est prévue pour
                    l'automne 2021.
            


            L'EBSI est en filiation directe avec l'École des bibliothécaires, fondée en 1937.
                Totalement restructurée en 1961, elle devient partie intégrante de l'Université de
                Montréal sous le nom d'École de bibliothéconomie. De 1961 à 1970, elle dispense un
                programme de baccalauréat en bibliothéconomie en 1 an (1961 à 1966) puis en 2 ans
                (1966 à 1971). À l'instar des autres écoles canadiennes et américaines, et pour se
                conformer aux exigences de l'agrément de l'American Library Association (ALA),
                l'École de bibliothéconomie dispense, à partir de la décennie 1970, un programme de
                maîtrise. Le programme, agréé par l'ALA, exige que les étudiants admis en maîtrise,
                premier grade terminal dans la discipline, possèdent à l'admission un premier cycle
                universitaire dans n'importe quel domaine du savoir, sciences, sciences sociales,
                sciences humaines, sciences de la santé, science politique, ingénierie, etc.
            L'École a obtenu l'agrément de l'ALA pour la première fois en 1969 et son agrément
                lui fut successivement renouvelé en 1976, 1985, 1992, 2000, 2008 et 2015. Cette
                évaluation en profondeur a valeur de reconnaissance de qualité pour l'École qui,
                ayant obtenu l'agrément, peut participer au réseau des écoles agréées
                nord-américaines. Pour ses diplômés, l'agrément ouvre le vaste marché de l'emploi
                relié aux champs disciplinaires des sciences de l'information au Québec, bien sûr,
                mais aussi dans le reste du Canada et aux États-Unis.
            De 1971 à 1984, l'École a décerné un diplôme de maîtrise en bibliothéconomie (M.
                Bibl.). En 1984, l'École a pris le nom d'École de bibliothéconomie et des sciences
                de l'information et le libellé du diplôme de maîtrise a été changé en maîtrise en
                bibliothéconomie et sciences de l'information (M.B.S.I.). En 1998, après une
                révision en profondeur du programme de maîtrise, le nouveau diplôme est devenu
                maîtrise en sciences de l'information (M.S.I.).
            L'École offre également, depuis 1983, un certificat en archivistique et depuis 2001,
                un certificat en gestion de l'information numérique. Les objectifs principaux du
                certificat en archivistique sont de permettre à l'étudiant d'acquérir des
                connaissances de base théoriques et pratiques en archivistique et de se familiariser
                avec les principes et les méthodes spécifiques aux archives vus sous l'angle de
                l'administration et de la recherche; de développer des habiletés pour concevoir,
                appliquer et utiliser des outils archivistiques ainsi que des attitudes propres à
                l'exercice des fonctions archivistiques. Le certificat en gestion de l'information
                numérique prépare l'étudiant à exercer des activités de gestion de l'information
                numérique par le développement de compétences sur les plans de la création, de
                l'organisation, de la préservation, de la recherche et de la diffusion de cette
                information. Il lui permettra d'évoluer avec aisance dans un environnement de
                travail hautement technologique.
            Enfin, depuis septembre 1997, l'École dispense un programme de doctorat en sciences
                de l'information.
        
    
    
        \newpage
        \section {
        Coordonnées
        }
        
        
        
        \textit{
        Guides > Renseignements généraux > Présentation de l'EBSI > Coordonnées
            - EBSI - Université de Montréal - Confluence
        }
    
        Coordonnées
        
            
        \textbf{
        Adresse géographique*
        }
    
            École de bibliothéconomie et des sciences de l'informationUniversité de
                Montréal, Pavillon Lionel-Groulx3150, rue Jean-Brillant, bur.
                C-2004Montréal (Québec) H3T 1N8
            
        \textbf{
        Adresse postale
        }
    
            École de bibliothéconomie et des sciences de l'informationUniversité de
                Montréal, Pavillon Lionel-GroulxC.P. 6128, succursale Centre-villeMontréal
                (Québec) H3C 3J7 Canada
            
        \textbf{
        Site web et adresse courriel
        }
    
            
                
        \href{
        http://www.ebsi.umontreal.ca
        } {
        http://www.ebsi.umontreal.ca
        }
    
            
            
                
        \href{
        mailto:ebsiinfo@ebsi.umontreal.ca
        } {
        ebsiinfo@ebsi.umontreal.ca
        }
    
            
            
        \textbf{
        Téléphone et télécopieur
        }
    
            Téléphone : +1 514 343-6044
            Télécopieur : +1 514 343-5753
        
        
            
            * Merci de ne pas envoyer de courrier postal à cette adresse.
        
    
    
        \newpage
        \section {
        Locaux
        }
        
        
        
        \textit{
        Guides > Renseignements généraux > Présentation de l'EBSI > Locaux -
            EBSI - Université de Montréal - Confluence
        }
    
        Locaux
        
            
                Pavillon Lionel-Groulx, 3150, rue Jean-Brillant,
                    2e étage
            
            
                
                    
                
            
        
    
    
        \newpage
        \section {
        Structure administrative
        }
        
        
        
        \textit{
        Guides > Renseignements généraux > Présentation de l'EBSI > Structure
            administrative - EBSI - Université de Montréal - Confluence
        }
    
        Structure administrative
        
            L'EBSI est dirigée par le directeur qui est nommé pour des mandats de quatre ans.
                L'assemblée de département se réunit normalement une fois tous les deux mois de
                septembre à mai. Les membres votants sont les professeurs réguliers, les
                représentants étudiants et le représentant des chargés de cours.
            La structure de l'ensemble des comités, incluant ceux qui n'ont pas de représentants
                étudiants ou représentant des chargés de cours, et la description de leur mandat se
                trouve ci-dessous.
            
                
                    
                
            
            
        \textbf{
        Composition des comités
        }
    
            
                
                    
                        
                             
                            Professeurs
                            Professionnels
                            Étudiants 1er cycle
                            Étudiants 2e cycle
                            Étudiants 3e cycle
                            Chargés de cours
                            Diplômés
                            Personnel admin.
                            Externes
                        
                        
                            Assemblée de département*
                            Tous
                            
                                Tous
                            
                            0
                            2
                            1
                            1‡
                            1
                            2
                            1
                        
                        
                            Comités de programmes
                        
                        
                            
        \href{
        https://wiki.umontreal.ca/pages/viewpage.action?pageId=124095459#coet
        } {
        Comité des études
        }
    *
                            4
                            1
                            1
                            1
                            1
                            1
                            1
                            1
                            0
                        
                        
                            
        \href{
        https://wiki.umontreal.ca/pages/viewpage.action?pageId=124095459#coetsup
        } {
        Comité d'études supérieures
        }
    *(incluant
                                admissions et bourses)
                            4
                            0
                            0
                            0
                            0
                            0
                            0
                            1
                            0
                        
                        
                            
                                
        \href{
        https://wiki.umontreal.ca/pages/viewpage.action?pageId=124095459#coprec
        } {
        Comité du premier cycle
        }
    
                            
                            3
                            0
                            0
                            0
                            0
                            0
                            0
                            1
                            0
                        
                        
                            
                                
        \href{
        https://wiki.umontreal.ca/pages/viewpage.action?pageId=124095459#coca
        } {
        Comité du certificat en archivistique
        }
    
                            
                            2
                            1
                            1
                            0
                            0
                            1
                            1
                            0
                            0
                        
                        
                            
                                
        \href{
        https://wiki.umontreal.ca/pages/viewpage.action?pageId=124095459#cogin
        } {
        Comité du certificat en gestion de l'information
                                    numérique
        }
    
                            
                            2
                            1
                            1
                            0
                            0
                            1
                            1
                            0
                            0
                        
                        
                            
                                
        \href{
        https://wiki.umontreal.ca/pages/viewpage.action?pageId=124095459#comsi
        } {
        Comité de la maîtrise
        }
    
                            
                            3
                            2
                            0
                            1
                            0
                            1
                            1
                            1
                            0
                        
                        
                            
                                
        \href{
        https://wiki.umontreal.ca/pages/viewpage.action?pageId=124095459#comed
        } {
        Comité des études doctorales
        }
    
                            
                            4
                            0
                            0
                            0
                            1
                            0
                            1
                            0
                            0
                        
                        
                            Comités consultatifs
                        
                        
                            
                                
        \href{
        https://wiki.umontreal.ca/pages/viewpage.action?pageId=124095459#comid
        } {
        Comité d'informatique documentaire
        }
    
                            
                            2
                            1
                            0
                            1
                            1
                            0
                            0
                            0
                            0
                        
                        
                            
                                
        \href{
        https://wiki.umontreal.ca/pages/viewpage.action?pageId=124095459#corech
        } {
        Comité de la recherche
        }
    
                            
                            3
                            0
                            0
                            0
                            1
                            0
                            0
                            0
                            0
                        
                        
                            
                                
                                    
        \href{
        https://wiki.umontreal.ca/pages/viewpage.action?pageId=124095459#cocomi
        } {
        Comité des conférences midi
        }
    
                                
                            
                            1
                            0
                            1
                            1
                            1
                            0
                            0
                            0
                            0
                        
                        
                            
                                
        \href{
        https://wiki.umontreal.ca/pages/viewpage.action?pageId=124095459#cach
        } {
        Comité d'accueil des chargés de cours
        }
    
                            
                            3
                            1
                            0
                            0
                            0
                            0
                            0
                            1
                            0
                        
                        
                            
                                
        \href{
        https://wiki.umontreal.ca/pages/viewpage.action?pageId=124095459#ceco
        } {
        Comité de l'évaluation continue
        }
    
                            
                            2
                            1
                            0
                            0
                            0
                            0
                            0
                            1
                            0
                        
                        
                            Comités externes
                        
                        
                            
                                
        \href{
        https://wiki.umontreal.ca/pages/viewpage.action?pageId=124095459#clip
        } {
        Comité local d'intégration pédagogique
        }
    
                            
                            2
                            0
                            0
                            0
                            0
                            2
                            0
                            0
                            0
                        
                        
                            
                                
        \href{
        https://wiki.umontreal.ca/pages/viewpage.action?pageId=124095459#ci
        } {
        Comité de divulgation des conflits d'intérêt
        }
    
                            
                            3
                            0
                            0
                            1†
                            1†
                            0
                            1
                            0
                            0
                        
                        
                            
                                
        \href{
        https://wiki.umontreal.ca/pages/viewpage.action?pageId=124095459#close
        } {
        Comité local de soutien à l'enseignement
        }
    
                                *
                            
                            2
                            1
                            0
                            0
                            0
                            0
                            0
                            0
                            0
                        
                        
                            
                                
        \href{
        https://wiki.umontreal.ca/pages/viewpage.action?pageId=124095459#codchap
        } {
        Comité départemental sur la charge
                                    professorale
        }
    
                            
                            3 à 5
                            0
                            0
                            0
                            0
                            0
                            0
                            0
                            0
                        
                    
                
            
            Légende :* Comités statutaires† Un (seul) étudiant
                des cycles supérieurs siège si le comité traite des dossiers d'étudiants ou de
                stagiaires postdoctoraux.‡ Un chargé de cours est requis uniquement si le
                nombre de charges de cours est supérieure ou égale à 10 par année.
            
        \textbf{
        Mandats des comités de l'EBSI
        }
    
            Au début de chaque année académique, la direction de l'EBSI établit la composition
                des divers comités.
            Comité des études (COÉT) (Comité statutaire la FAS)
            Les Statuts de l'université prévoient à l'article 34.01 qu'« un comité des
                études est constitué dans les départements ». Le Comité des études donne son
                avis à l'assemblée de département sur tout projet d'élaboration, de modification de
                programme d'études, de cours et autres activités pédagogiques et de modification du
                règlement pédagogique. À l'EBSI, le COÉT reçoit l'avis de quatre comités, le Comité
                de la maîtrise en sciences de l'information, le Comité du certificat en
                archivistique, le Comité du certificat en gestion de l'information numérique, et le
                Comité des études doctorales, qu'il autorise à donner directement leur avis à
                l'Assemblée de département relativement au programme qui les concerne. Le COÉT est
                un comité statutaire de la FAS et sa composition est régie par le règlement de la
                FAS; pour l'EBSI la composition est la suivante : Directeur + 3 étudiants (dont
                au moins 1 étudiant des cycles supérieurs) + 3 professeurs + 2 diplômés. En principe
                la durée des mandats des membres ne devrait pas dépasser trois ans.
            Comité des études supérieures (COÉTSUP) (Comité statutaire de la
                FESP)
            Ce comité, prévu au règlement pédagogique de la FESP, doit examiner, à la fin de
                chaque trimestre ou de chaque année, le dossier de tous les étudiants inscrits à un
                programme d'études supérieures et avise par écrit les étudiants qu'il considère à
                risque d'échec. Le rôle du Comité des études supérieures en est surtout un
                d'arbitrage en cas de litige de nature pédagogique. Il doit faire rapport à
                l'assemblée de département au moins une fois par année. Plus spécifiquement,
                l'article 31 du Règlement pédagogique de la Faculté des études supérieures et
                postdoctorales précise que :
            Le comité d'études supérieures effectue le suivi des
                étudiants selon les dispositions pertinentes du règlement pédagogique de la Faculté
                des études supérieures et postdoctorales et s'assure que les objectifs de formation
                sont partagés par les membres du corps professoral et les étudiants et que la
                manière de les atteindre s'appuie sur des pratiques favorisant la réussite des
                études.
            De plus, les responsabilités du comité d'études supérieures peuvent comprendre :
            
        \begin{itemize}
        
                
        \item le recrutement, l'admission et l'inscription des étudiants;
                
        \item l'application des politiques locales de soutien financier aux étudiants;
                
        \item l'accueil des nouveaux étudiants et leur intégration aux milieux de
                    formation;
                
        \item l'évaluation des étudiants.
            
        \end{itemize}
    
            En vertu de l'article 35, le comité d'études supérieures examine aussi les demandes
                de suspension ou de prolongation de scolarité et peut « recommander au doyen
                d'autoriser l'étudiant à s'inscrire au trimestre suivant, de mettre fin à la
                candidature de celui-ci — selon les articles 59g) et 88i) — pour le doctorat ou de
                lui accorder un trimestre de probation ».
            La composition du comité (au moins trois professeurs) est déterminée par le règlement
                pédagogique de la FESP.
            Comité des admissions (COMAD)
            Sous-comité du Comité des études supérieures, le Comité des admissions étudie toutes
                des demandes d'admission (régulières et en programmes d'échanges) au programme de
                maîtrise en sciences de l'information et recommande ou non l'admission des
                candidats. Il voit à l'élaboration et à la mise à jour de tous les documents et
                outils requis pour l'admission et la sélection des candidats. Les membres du COMAD
                sont en principe choisis parmi les membres du COÉTSUP.
            Comité des bourses (COBO)
            Sous-comité du Comité des études supérieures, le Comité des bourses a pour mandat
                l'octroi des diverses bourses d'étude disponibles aux étudiants de l'École. Il fait
                la promotion des concours, voit à l'application des règles propre à chaque concours,
                reçoit et analyse les dossiers des candidats, sélectionne les récipiendaires et
                organise une séance annuelle de remise des bourses, excluant les bourses de voyage
                réservées aux étudiants au doctorat qui sont attribuées par les membres du COMED.
                Les membres du COBO sont en principe choisis parmi les membres du COÉTSUP.
            Comité du premier cycle (COPREC)
            Le rôle du Comité du premier cycle en est surtout un de coordination entre les deux
                comités des certificats. Le comité doit examiner, à la fin de chaque trimestre ou de
                chaque année, le dossier de tous les étudiants inscrits aux certificats et avise par
                écrit les étudiants qu'il considère à risque d'échec.
            Comité de la maîtrise en sciences de l'information (COMSI)
            Sous-comité du COÉT présidé par le directeur de l'EBSI, le Comité de la maîtrise
                donne son avis sur l'élaboration, les modifications du programme, des cours et des
                activités pédagogiques de maîtrise et sur les modifications du Règlement pédagogique
                du programme. Il approuve l'enregistrement des sujets de recherche des étudiants
                inscrits à l'orientation recherche. Il participe également à la gestion et à
                l'animation du programme et participe à la promotion du programme de concert avec le
                Comité des relations publiques. Il produit un rapport bilan à la fin de chaque année
                académique qu'il présente à l'Assemblée de département.
            Comité du certificat en archivistique (COCA)
            Sous-comité du COÉT présidé par le responsable du programme du certificat en
                archivistique, le Comité du certificat en archivistique donne son avis sur
                l'élaboration, les modifications du programme, des cours et des activités
                pédagogiques du certificat. Il anime et coordonne les activités du certificat et
                participe à la promotion du programme de concert avec le Comité des relations
                publiques. Il produit un rapport bilan à la fin de chaque année académique qu'il
                présente à l'Assemblée de département.
            Comité du certificat en gestion de l'information numérique (COGIN)
            Sous-comité du COÉT et présidé par le responsable du programme du certificat en
                gestion de l'information numérique, le Comité du certificat en gestion de
                l'information numérique donne son avis sur l'élaboration, l'évaluation et les
                modifications du programme, des cours et des activités pédagogiques du certificat.
                Il anime et coordonne les activités du certificat et participe à la promotion du
                programme de concert avec le Comité des relations publiques. Il produit un rapport
                bilan à la fin de chaque année académique qu'il présente à l'Assemblée de
                département.
            Comité des études doctorales (COMED)
            Sous-comité du COÉT présidé par le coordonnateur des études doctorales, le Comité des
                études doctorales donne son avis sur l'élaboration, les modifications du programme,
                des cours et des activités pédagogiques de doctorat et sur les modifications du
                Règlement pédagogique du programme. Il approuve la formation des comités de
                recherche et l'enregistrement des sujets de recherche de doctorants. Il voit à la
                composition des jurys de l'examen général de synthèse. Il participe également à la
                gestion et à l'animation du programme, sélectionne les candidats et participe au
                recrutement des candidats et à la promotion du programme de concert avec le Comité
                des relations publiques. Il produit un rapport bilan à la fin de chaque année
                académique qu'il présente à l'Assemblée de département.
            Comité d'informatique documentaire (COMID)
            Le Comité d'informatique documentaire voit à établir et à tenir à jour un plan de
                développement de l'informatique et des technologies de l'information à l'École. Il
                dresse chaque année l'état des besoins en équipements, logiciels et ressources web
                pour l'enseignement et la recherche à l'École. Il fait rapport à l'Assemblée de
                département sur toute question relative à l'informatique documentaire et aux
                technologies de l'information.
            Comité de la recherche (CORECH)
            Le Comité de la recherche voit à stimuler et animer la recherche à l'École, à
                recueillir et à diffuser aux professeurs toute l'information pertinente sur la
                recherche. Il participe à la promotion de la recherche à l'EBSI de concert avec le
                Comité des relations publiques.
            Comité des conférences-midi (COCOMI)
            Sous-comité du Comité de la recherche, le Comité des conférences midi voit à la
                programmation et à l'organisation des conférences midi de l'École. Il est également
                responsable de la promotion et de l'évaluation des conférences midi. Le COCOMI est
                un comité conjoint de l'École et de l'AEEEBSI. Les membres du COCOMI sont en
                principe choisis parmi les membres du CORECH.
            Comité local de soutien à l'enseignement (CLoSE)
            Le comité est consulté sur l'élaboration de la politique sur le soutien à
                l'enseignement (attribution d'aide aux enseignants) et font des recommandations au
                département concernant l'application de la politique de soutien à l'enseignement.
                Ils s'assurent que la politique sur le soutien à l'enseignement est publique et
                accessible. Le comité reçoit les plaintes de chargées et chargés de cours relatives
                à la répartition des auxiliariats d'enseignement et s'assurent de l'application de
                la convention. Le comité en fait rapport à la direction.
            Comité d'accueil des chargés de cours (CACH)
            Ce comité est en charge de l'accueil des nouveaux chargés de cours afin de s'assurer
                que toute l'information pertinente leur soit transmise. Les membres du comité
                s'assureront de fournir des réponses adéquates aux questions des nouveaux chargés de
                cours.
            Comité de l'évaluation continue (CECO)
            Le mandat de cette instance consiste à implanter à l'EBSI un processus d'évaluation
                continue des programmes. À cette fin, le CECO :
            
        \begin{itemize}
        
                
        \item Tiendra les sessions de formation et d'information nécessaires à la réalisation
                    de son mandat;
                
        \item Compilera et analysera les renseignements qui lui auront été transmis
                    annuellement lors de l'application de la démarche d'évaluation continue des
                    programmes;
                
        \item Élaborera et tiendra à jour, sous forme de Tableau de bord, une base de données
                    contenant toutes les informations pertinentes à l'évaluation continue des
                    programmes;
                
        \item Élaborera les documents et autres outils d'analyse et de reddition de compte
                    qu'il jugera utile à l'implantation, au suivi et à l'évaluation de la démarche
                    d'évaluation continue des programmes;
                
        \item Produira les rapports détaillés ou synthèses utiles, notamment en prévision de
                    la réunion bilan annuelle.
            
        \end{itemize}
    
            Comité local d'intégration pédagogique (CLIP)
            
                « [L]e mandat des Comités locaux d'intégration pédagogiques est de :
                
        \begin{itemize}
        
                    
        \item Préparer et adopter un plan annuel de fonctionnement pour leur unité et un
                        plan de priorités pédagogiques, s'il y a lieu.
                    
        \item Favoriser la réalisation de projets pédagogiques dans leur unité.
                    
        \item Évaluer les projets qui lui sont soumis, en lien avec les objectifs
                        départementaux, facultaires et institutionnels.
                    
        \item Acheminer au comité universitaire d'intégration pédagogique tous les projets
                        reçus accompagnés de recommandations, favorables ou non.
                    
        \item Acheminer, pour information, à l'Assemblée départementale ou à l'Assemblée
                        de Faculté, les projets qu'il recommande.
                    
        \item Déposer, au début de l'année universitaire, la planification de ses
                        activités à l'instance académique désignée par le (la) responsable de
                        l'unité.
                    
        \item Présenter un bilan annuel à l'unité académique indiquant les réalisations et
                        l'état d'avancement des projets en cours.
                    
        \item Déposer le bilan annuel de ses activités au Comité universitaire
                        d'intégration pédagogique.
                    
        \item Au besoin, faire des rencontres avec les associations étudiantes pour les
                        informer des projets adoptés dans leur unité académique.
                    
        \item Travailler en collaboration avec le Comité des études de l'unité. »
                
        \end{itemize}
    
            
            Source : 
        \href{
        http://scccum.ca/wp-content/uploads/2014/06/D_Infomation-CUIP.pdf
        } {
        http://scccum.ca/wp-content/uploads/2014/06/D_Infomation-CUIP.pdf
        }
    
            Comité de divulgation des conflits d'intérêt (CODICOIN)
            Ce comité a comme responsabilité de vérifier et de valider les déclarations de
                conflit d'intérêt tel que requis par le rectorat de l'université.
            Comité départemental sur la charge professorale (CODCHAP)
            Au plus tard le 1er octobre, le comité propose au
                directeur de l'unité, dans le respect des balises prévues à l'annexe VIII de la
                convention collective, une charge de travail normale pour l'unité. Cette proposition
                doit inclure l'ensemble des données factuelles reçues et est simultanément déposée à
                l'assemblée départementale.
             
        
    
    
        \newpage
        \section {
        Personnel
        }
        
        
        
        \textit{
        Guides > Renseignements généraux > Personnel - EBSI - Université de Montréal -
            Confluence
        }
    
        Personnel
        
            
        \textbf{
        Professeurs
                réguliers
        }
    
            Arsenault, Clément (professeur agrégé, directeur, responsable
                de la maîtrise en sciences de l'information et responsable par intérim du certificat
                en gestion de l'information numérique)Téléphone : 514 343-7400 / Bureau :
                    C-2012
        \href{
        mailto:clement.arsenault@umontreal.ca
        } {
        clement.arsenault@umontreal.ca
        }
    
            Bergeron, Pierrette (professeure agrégée)Téléphone : 514
                343-5651 / Bureau : C-2022
        \href{
        mailto:pierrette.bergeron@umontreal.ca
        } {
        pierrette.bergeron@umontreal.ca
        }
    
            Da Sylva, Lyne (professeure agrégée)Téléphone : 514
                343-6444 / Bureau : C-2030
        \href{
        mailto:lyne.da.sylva@umontreal.ca
        } {
        lyne.da.sylva@umontreal.ca
        }
    
            Demoulin, Marie (professeure adjointe)Téléphone : 514
                343-6111, poste 0938 / Bureau : C-2052
        \href{
        mailto:marie.demoulin@umontreal.ca
        } {
        marie.demoulin@umontreal.ca
        }
    
            Desrochers, Nadine (professeure adjointe)Téléphone : 514
                343-6111, poste 1290 / Bureau : C-2018
        \href{
        mailto:nadine.desrochers@umontreal.ca
        } {
        nadine.desrochers@umontreal.ca
        }
    
            Dufour, Christine (professeure agrégée)Téléphone : 514
                343-6111, poste 4164 / Bureau : C-2072
        \href{
        mailto:christine.dufour@umontreal.ca
        } {
        christine.dufour@umontreal.ca
        }
    
            Forest, Dominic (professeur agrégé)Téléphone : 514
                343-6119 / Bureau : C-2046
        \href{
        mailto:dominic.forest@umontreal.ca
        } {
        dominic.forest@umontreal.ca
        }
    
            Laplante, Audrey (professeure agrégée, responsable par intérim
                du doctorat en sciences de l'information)Téléphone : 514 343-6048 / Bureau :
                    C-2020
        \href{
        mailto:audrey.laplante@umontreal.ca
        } {
        audrey.laplante@umontreal.ca
        }
    
            Larivière, Vincent (professeur agrégé, responsable de la
                    maîtrise en sciences de l'information - recherche)Téléphone : 514
                343-5600 / Bureau : C-2038
        \href{
        mailto:vincent.lariviere@umontreal.ca
        } {
        vincent.lariviere@umontreal.ca
        }
    
            Lemay, Yvon (professeur agrégé, responsable du certificat en
                archivistique)Téléphone : 514 343-7552 / Bureau : C-2042
        \href{
        mailto:yvon.lemay@umontreal.ca
        } {
        yvon.lemay@umontreal.ca
        }
    
            Leroux, Éric (professeur agrégé)Téléphone : 514 343-6071 /
                Bureau : C-2048
        \href{
        mailto:eric.leroux@umontreal.ca
        } {
        eric.leroux@umontreal.ca
        }
    
            Marcoux, Yves (professeur agrégé)Téléphone : 514 343-7750
                / Bureau : C-2044
        \href{
        mailto:yves.marcoux@umontreal.ca
        } {
        yves.marcoux@umontreal.ca
        }
    
            Mas, Sabine (professeure agrégée)Téléphone : 514 343-2454
                / Bureau : C-2040
        \href{
        mailto:sabine.mas@umontreal.ca
        } {
        sabine.mas@umontreal.ca
        }
    
            Maurel, Dominique (professeure agrégée)Téléphone : 514
                343-7204 / Bureau : C-2028
        \href{
        mailto:dominique.maurel@umontreal.ca
        } {
        dominique.maurel@umontreal.ca
        }
    
            Savard, Réjean (professeur titulaire)Téléphone : 514
                343-7408 / Bureau : C-2036
        \href{
        mailto:rejean.savard@umontreal.ca
        } {
        rejean.savard@umontreal.ca
        }
    
            
        \textbf{
        Professeurs
                associés
        }
    
            Gagnon-Arguin, Louise (professeure associée)Bureau :
                    C-2007
        \href{
        mailto:louise.gagnon-arguin@umontreal.ca
        } {
        louise.gagnon-arguin@umontreal.ca
        }
    
            Hudon, Michèle (professeure associée)Bureau : C-2007
        \href{
        mailto:michele.hudon@umontreal.ca
        } {
        michele.hudon@umontreal.ca
        }
    
            Lajeunesse, Marcel (professeur associé)Bureau :
                    C-2007
        \href{
        mailto:marcel.lajeunesse@umontreal.ca
        } {
        marcel.lajeunesse@umontreal.ca
        }
    
            Turner, James (professeur associé)
        \href{
        mailto:james.turner@umontreal.ca
        } {
        james.turner@umontreal.ca
        }
    
            
        \textbf{
        Personnel
                administratif
        }
    
            Boyle, Brigitte (adjointe au directeur)Téléphone : 514
                343-7406 / Bureau : C-2008
        \href{
        mailto:brigitte.boyle@umontreal.ca
        } {
        brigitte.boyle@umontreal.ca
        }
    
            Lessard, Julie (technicienne en coordination du travail de
                bureau; technicien à la gestion des dossiers étudiants,
                1er cycle)Téléphone : 514 343-6646 / Bureau : C-2004
        \href{
        mailto:julie.lessard.5@umontreal.ca
        } {
        julie.lessard.5@umontreal.ca
        }
     
            Pasutto, Sarah (technicienne en coordination du travail de
                bureau – direction)Téléphone : 514 343-6111, poste 5103 / Bureau :
                    C-2010
        \href{
        mailto:sarah.pasutto@umontreal.ca
        } {
        sarah.pasutto@umontreal.ca
        }
    
            Tremblay, Alain (technicien à la gestion des dossiers
                étudiants, études supérieures)Téléphone : 514 343-6044 / Bureau : C-2004
        \href{
        mailto:alain.tremblay.1@umontreal.ca
        } {
        alain.tremblay.1@umontreal.ca
        }
    
            
        \textbf{
        Personnel
                professionnel
        }
    
            Bélanger, Martin (responsable de formation
                professionnelle)Téléphone : 514 343-6111, poste 1743 / Bureau : C-2023
        \href{
        mailto:martin.belanger.5@umontreal.ca
        } {
        martin.belanger.5@umontreal.ca
        }
    
            Bourgey, Isabelle (coordonnatrice de stages)Téléphone :
                514 343-2243 / Bureau : C-2024
        \href{
        mailto:isabelle.bourgey@umontreal.ca
        } {
        isabelle.bourgey@umontreal.ca
        }
    
            d'Alayer, Arnaud (responsable des laboratoires d'informatique
                documentaire)Téléphone : 514 343-6111, poste 1040 / Bureau : C-2034
        \href{
        mailto:arnaud.dalayer@umontreal.ca
        } {
        arnaud.dalayer@umontreal.ca
        }
    
            Dion, Isabelle (coordonnatrice de stages – responsable du
                laboratoire d'archivistique)Téléphone : 514 343-2244 / Bureau : C-2026
        \href{
        mailto:isabelle.dion@umontreal.ca
        } {
        isabelle.dion@umontreal.ca
        }
    
            Maatallah, Mohammed (administrateur de systèmes)Téléphone
                : 514 343-2246 / Bureau : C-2025
        \href{
        mailto:mohamed.maatallah@umontreal.ca
        } {
        mohamed.maatallah@umontreal.ca
        }
    
            Trinh, Minh Thi (conseillère en informatique
                documentaire)Téléphone : 514 343-6111, poste 1292 / Bureau : C-2035-1
        \href{
        mailto:minh.thi.trinh@umontreal.ca
        } {
        minh.thi.trinh@umontreal.ca
        }
    
        
    
    
        \newpage
        \section {
        Portail de l'Université de Montréal
        }
        
        
        
        \textit{
        Guides > Soutien à l'enseignement > Portail de l'Université de
            Montréal - EBSI - Université de Montréal - Confluence
        }
    
        Portail de l'Université de Montréal
        
            Mon portail UdeM est une application web offrant un guichet unique regroupant les
                informations et applications institutionnelles. Cette application est accessible en
                tout temps et offre aux utilisateurs un accès sécuritaire aux informations et
                services institutionnels en fonction de leurs rôles à l'Université. L'adresse
                    du portail est 
        \href{
        http://www.portail.umontreal.ca/
        } {
        http://www.portail.umontreal.ca/
        }
    .
                    Vous devez vous identifier sur la page d'authentification avec votre code
                    d'accès et votre UNIP / mot de passe.
            
        \textbf{
        Étudiants
        }
    
            Mon portail UdeM vous permet de gérer votre dossier étudiant en accédant à votre
                dossier académique via le Centre étudiant, à votre profil informatique à la DGTIC, à
                l’horaire détaillé des cours auxquels vous êtes inscrit, etc.
            
        \textbf{
        Employés
        }
    
            Diverses fonctions sont disponibles à travers le portail qui permet, par exemple,
                d'accéder à votre dossier à la Direction des ressources humaines (Synchro RH), à
                votre dossier Centre Corps Professoral (Synchro Académique), à StudiUM, à votre
                profil informatique à la DGTIC, aux ressources offertes par divers services de
                l'Université, etc.
        
    
    
        \newpage
        \section {
        Accéder aux services informatiques (1er, 2e et 3e cycles)
        }
        
        
        
        \textit{
        Guides > Soutien à l'enseignement > Accéder aux services informatiques
            (1er, 2e et 3e cycles) - EBSI - Université de Montréal - Confluence
        }
    
        Accéder aux services informatiques (1er, 2e et 3e cycles)
        
            Pour accéder aux différents services informatiques offerts par l'Université de
                Montréal (dont l'application Mon portail UdeM, au Centre étudiant, à
                StudiUM, au réseau sans fil, au serveur de courriel de l'Université,
                etc.) et de l’EBSI (postes de travail des laboratoires d'informatique, à
                votre espace sur le serveur GIN-EBSI, etc.), vous devrez vous authentifier.
            Pour ce faire, vous devez utiliser votre code d'accès et votre UNIP / mot de
                    passe.Le code d'accès, aussi nommé « code d'accès
                DGTIC », « code d'identification », « code d'usager », « nom d'utilisateur
                » ou « login », aura généralement la forme
                    p0123456. Votre UNIP (pour Université de Montréal – Numéro
                d'identification personnel), qui sera votre mot de passe (aussi parfois
                nommé « mot de passe SIM »).
            
                Le code d'accès, ainsi qu'un UNIP temporaire sont acheminés au candidat par
                    courriel lors de l'admission.
                Avant de pouvoir utiliser les ressources
                    informatiques de l'Université, vous devez modifier votre UNIP temporaire, qui
                    vous sera transmis par courriel ou téléphone. Pour ce faire, accédez à Mon
                    portail UdeM à l'adresse 
                
        \href{
        http://www.portail.umontreal.ca/
        } {
        http://www.portail.umontreal.ca/
        }
    
                , rubrique UNIP / mot de passe – Modifier.
                    Assurez-vous de respecter les règles d'écriture mentionnées pour que votre UNIP
                    / mot de passe soit valide.
            
            Le matricule (aussi nommé « matricule Synchro » ou « matricule étudiant »), est
                l'identifiant correspond à votre numéro de dossier étudiant à l'Université de
                Montréal. Vous pouvez connaître votre matricule en accédant au site Mon portail
                UdeM, rubrique « Centre étudiant – Données personnelles ». Retenez-le, car il vous
                sera utile lors de vos communications avec votre technicien en gestion de dossier
                étudiant et comme référence pour les paiements bancaires en ligne de vos frais de
                scolarité. C'est également cet identifiant que vous devez indiquer sur la page de
                présentation des travaux écrits que vous remettez.
        
    
    
        \newpage
        \section {
        Adresse de courriel (1er, 2e et 3e cycles)
        }
        
        
        
        \textit{
        Guides > Soutien à l'enseignement > Adresse de courriel (1er, 2e et 3e
            cycles) - EBSI - Université de Montréal - Confluence
        }
    
        Adresse de courriel (1er, 2e et 3e cycles)
        
            Une adresse de courriel ayant généralement la forme
                    prénom.nom@umontreal.ca vous sera attribuée lors de votre
                inscription. Ce sera votre adresse officielle de courriel à l'Université. Vous
                pouvez vérifier votre adresse de courriel en accédant au site Mon portail UdeM,
                rubrique « Mon profil — Courriel institutionnel ». Plus d'information à ce sujet
                vous sera communiquée en début de semestre.
        
    
    
        \newpage
        \section {
        Laboratoires d'informatique documentaire de l'EBSI
        }
        
        
        
        \textit{
        Guides > Soutien à l'enseignement > Laboratoires d'informatique
            documentaire de l'EBSI - EBSI - Université de Montréal - Confluence
        }
    
        Laboratoires d'informatique documentaire de l'EBSI
        
            Les laboratoires d'informatique de 1er et 2e cycles (C-2031 et
                C-2035 du pavillon Lionel-Groulx) sont mis à la disposition des étudiants inscrits
                aux programmes offerts à l'EBSI et du personnel de l'EBSI. La salle des étudiants du
                doctorat (C-2003) est réservée exclusivement à ceux-ci. Il est à noter que les
                travaux pratiques du certificat en gestion de l'information numérique se tiennent
                dans les 
        \href{
        /pages/viewpage.action?pageId=124095712
        } {
        Laboratoires
                    d'informatique facultaires
        }
    .
            Les utilisateurs des laboratoires d'informatique de l'EBSI s'engagent à respecter les
                        
        \href{
        http://www.ebsi.umontreal.ca/ressources-services/laboratoires-informatique-documentaire/EBSI-labo-info-politique-2013.pdf
        } {
        Règles en vigueur dans les laboratoires
                        d'informatique documentaire
        }
     ainsi que la 
        \href{
        http://secretariatgeneral.umontreal.ca/fileadmin/user_upload/secretariat/doc_officiels/reglements/administration/Ges40_28-politique-securite-informatique-utilisation-ressources-informatiques-universite-de-montreal.pdf
        } {
        Politique de sécurité et d'utilisation
                        des ressources informatiques de l'Université de
                    Montréal
        }
    . Ces règles, de même que l'horaire d'ouverture des
                laboratoires et divers documents relatifs aux ressources peuvent être
                consultées sur le site des laboratoires à l'adresse 
        \href{
        http://www.ebsi.umontreal.ca/ressources-services/laboratoires-informatique-documentaire/
        } {
        http://www.ebsi.umontreal.ca/ressources-services/laboratoires-informatique-documentaire/
        }
    .
            
                Responsable : Arnaud d'Alayer
        \href{
        mailto:arnaud.dalayer@umontreal.ca
        } {
        arnaud.dalayer@umontreal.ca
        }
    ,
                        514 343-6111, poste 1040
            
            
        \textbf{
        Assistance dans les laboratoires d'informatique
        }
    
            En dehors des périodes de travaux pratiques dirigés, les étudiants doivent réaliser
                leurs travaux sur ordinateur de façon autonome, et ce, même s'ils ne se sont pas
                présentés aux périodes de TP supervisés.
            Par ailleurs, compte tenu du grand nombre de logiciels utilisés dans le cadre des
                divers programmes d'enseignement et de la diversité des applications installées dans
                les laboratoires d'informatique, les objectifs visés par le personnel des
                laboratoires sont :
            
        \begin{itemize}
        
                
        \item D'installer sur les postes de travail des laboratoires et de configurer
                    adéquatement l'ensemble des logiciels requis pour les cours des divers
                    programmes de 1er et 2e cycles offerts à l'EBSI.
                
        \item D'assurer l'encadrement et le dépannage pour les applications générales, tout en
                    amenant les étudiants à résoudre leurs problèmes de façon autonome.
            
        \end{itemize}
    
            En ce qui a trait aux questions relatives aux logiciels spécialisés utilisés dans le
                cadre des cours, la personne responsable du cours (professeur ou chargé de cours)
                est la personne-ressource à laquelle les étudiants doivent se référer pour obtenir
                de l'aide.
        
    
    
        \newpage
        \section {
        Domaine cours.ebsi.umontreal.ca
        }
        
        
        
        \textit{
        Guides > Soutien à l'enseignement > Domaine cours.ebsi.umontreal.ca - EBSI
            - Université de Montréal - Confluence
        }
    
        Domaine cours.ebsi.umontreal.ca
        
            Les plans de cours et sites de cours des divers programmes offerts à l'EBSI sont
                accessibles à partir d'une même page web à l'adresse 
        \href{
        http://cours.ebsi.umontreal.ca
        } {
        http://cours.ebsi.umontreal.ca
        }
    . On y
                trouve des liens tant vers les sites web des cours en accès public qui résident sur
                le serveur GIN-EBSI que vers les contenus de cours disponibles dans StudiUM. De
                plus, les horaires de cours y sont présentés.
        
    
    
        \newpage
        \section {
        StudiUM (1er, 2e et 3e cycles)
        }
        
        
        
        \textit{
        Guides > Soutien à l'enseignement > StudiUM (1er, 2e et 3e cycles) - EBSI
            - Université de Montréal - Confluence
        }
    
        StudiUM (1er, 2e et 3e cycles)
        
            StudiUM est l'environnement numérique d'apprentissage de l'Université de Montréal.
                Plusieurs enseignants utilisent cet environnement pour la diffusion de leurs
                contenus de cours sur le web. Tout étudiant qui a un code d'accès et un UNIP / Mot
                de passe valide peut accéder aux contenus des cours auxquels il est inscrit, et qui
                sont disponibles dans StudiUM, à partir du site web de StudiUM (
        \href{
        http://studium.umontreal.ca
        } {
        http://studium.umontreal.ca
        }
    ).
        
    
    
        \newpage
        \section {
        Serveur GIN-EBSI
        }
        
        
        
        \textit{
        Guides > Soutien à l'enseignement > Serveur GIN-EBSI - EBSI - Université
            de Montréal - Confluence
        }
    
        Serveur GIN-EBSI
        
            Le serveur GIN-EBSI, géré par l'EBSI, est utilisé comme outil technologique en appui
                à l'ensemble des programmes d'enseignement offerts à l'EBSI.
            L'offre de service du serveur GIN-EBSI peut être consultée à l'adresse 
        \href{
        http://www.gin-ebsi.umontreal.ca/offre-service/
        } {
        http://www.gin-ebsi.umontreal.ca/offre-service/
        }
    .
            
        \textbf{
        Espace
                personnel
        }
    
            L'EBSI offre aux étudiants, professeurs et chargés de cours de ses divers programmes
                un espace de 100 Mo sur le serveur GIN-EBSI (
        \href{
        http://www.gin-ebsi.umontreal.ca/
        } {
        http://www.gin-ebsi.umontreal.ca/
        }
    ).
            Un compte sur le serveur GIN-EBSI vous sera donc attribué en début de trimestre. Les
                modalités d'accès à ce serveur, à partir du campus ou de la maison, sont présentées
                aux nouveaux étudiants lors d'une séance d'information. Ces modalités sont aussi
                accessibles à l'adresse 
        \href{
        http://www.gin-ebsi.umontreal.ca/
        } {
        http://www.gin-ebsi.umontreal.ca/
        }
    , sous
                la rubrique Accéder à mon espace du menu de navigation.
            
                Les comptes des étudiants, professeurs,
                    chargés de cours et auxiliaires d'enseignement sur le serveur GIN-EBSI sont
                    soumis aux règles et modalités d'utilisation décrites dans la
                        page Règles d'utilisation de l'espace de travail qui peut être
                    consultée à l’adresse 
        \href{
        http://www.gin-ebsi.umontreal.ca/regles-utilisation/
        } {
        http://www.gin-ebsi.umontreal.ca/regles-utilisation/
        }
    .
            
            
                Responsable : Mohamed Maatallah
        \href{
        mailto:mohamed.maatallah@umontreal.ca
        } {
        mohamed.maatallah@umontreal.ca
        }
    , 514 343-2246
            
        
    
    
        \newpage
        \section {
        Portfolio
        }
        
        
        
        \textit{
        Guides > Soutien à l'enseignement > Portfolio - EBSI - Université de
            Montréal - Confluence
        }
    
        Portfolio
        
            Portfolio (
        \href{
        http://portfolio.umontreal.ca/
        } {
        http://portfolio.umontreal.ca
        }
    ) est un nouvel environnement
                numérique implanté à l'UdeM basé sur le logiciel libre Mahara pour la gestion des
                portfolios électroniques et le réseautage social. Chaque membre de l'université
                (étudiants, enseignants, etc.) y a un compte auquel il accède à l'aide de ses
                identifiants UdeM. Dans cet environnement, les étudiants ont un rôle bien différent
                de celui qu'ils endossent dans StudiUM, soit un rôle de créateur de ressources
                numériques et de gestionnaire de cette information dans un environnement
                collaboratif. Portfolio vous permet entre autre de définir un profil numérique,
                utiliser un espace de stockage de 500 Mo, créer des groupes pour partager des
                ressources, créer des pages pour partager des contenus, communiquer à l'aide de
                forums de discussion, et créer un journal de bord en lien avec vos activités
                d'apprentissage.
        
    
    
        \newpage
        \section {
        Laboratoire d'archivistique (1er, 2e et 3e cycles)
        }
        
        
        
        \textit{
        Guides > Soutien à l'enseignement > Laboratoire d'archivistique (1er,
            2e et 3e cycles) - EBSI - Université de Montréal - Confluence
        }
    
        Laboratoire d'archivistique (1er, 2e et 3e cycles)
        
            Le laboratoire d'archivistique (salle C-2054) est un lieu pour compléter le cadre
                théorique de l'enseignement. Il offre aux enseignants la possibilité d'illustrer les
                notions archivistiques au programme, et permet aux étudiants de vivre une expérience
                pratique en milieu pédagogique.
            Le laboratoire sert à l'organisation et au traitement des archives, tant à des fins
                administratives que de recherche, et donne l'occasion de faire connaître et
                d'utiliser divers types de ressources documentaires : guides de classification,
                calendriers de conservation, politiques et procédures de gestion des archives,
                politiques d'acquisition, instruments de recherche, etc.
            L'horaire d'ouverture et la politique d'utilisation du laboratoire d'archivistique
                sont diffusés sur la page web du laboratoire d'archivistique (
        \href{
        http://www.ebsi.umontreal.ca/ressources-services/laboratoire-archivistique/
        } {
        http://www.ebsi.umontreal.ca/ressources-services/laboratoire-archivistique/
        }
    ).
            
                Responsable : Isabelle Dion
        \href{
        mailto:isabelle.dion@umontreal.ca
        } {
        isabelle.dion@umontreal.ca
        }
    , 514
                    343-2244
            
        
    
    
        \newpage
        \section {
        Laboratoires d'informatique facultaires
        }
        
        
        
        \textit{
        Guides > Soutien à l'enseignement > Laboratoires d'informatique
            facultaires - EBSI - Université de Montréal - Confluence
        }
    
        Laboratoires d'informatique facultaires
        
            La Faculté des arts et des sciences (FAS) met deux salles d'enseignement informatique
                à la disposition des professeurs et étudiants des départements situés dans le
                pavillon Lionel-Groulx, dont ceux de l'EBSI. Ces laboratoires, situés aux salles
                B-1215 (pavillon Jean-Brillant) et C-3115 (pavillon Lionel-Groulx) sont
                indépendants des laboratoires de l'EBSI mais le même type de règles d'utilisation y
                est en vigueur. Notez que les travaux pratiques des cours du certificat en gestion
                de l'information numérique se déroulent dans les laboratoires facultaires, des
                travaux pratiques des programmes de maîtrise en sciences de l'information ou du
                certificat en archivistique se déroulant aux mêmes heures dans les laboratoires
                d'informatique de l'EBSI.
            L'horaire d'ouverture et la politique d'utilisation des laboratoires facultaires
                peuvent être consultés à l'adresse 
        \href{
        http://fas.umontreal.ca/laboratoires/laboratoires-lionel-groulx/
        } {
        http://fas.umontreal.ca/laboratoires/laboratoires-lionel-groulx/
        }
    .
            
                Responsable : Djouher Hanifi
        \href{
        mailto:djouher.hanifi@umontreal.ca
        } {
        djouher.hanifi@umontreal.ca
        }
    , 514
                    343-6111 #43509
            
        
    
    
        \newpage
        \section {
        Logithèque de l'Université de Montréal
        }
        
        
        
        \textit{
        Guides > Aspects informatiques : services de la DGTIC > Logithèque de
            l'Université de Montréal - EBSI - Université de Montréal - Confluence
        }
    
        Logithèque de l'Université de Montréal
        
            
                « La DGTIC met à la disposition des membres de la communauté universitaire
                    différents logiciels qui ont fait l'objet d'ententes avec différents
                    fournisseurs. Selon les ententes, ces logiciels sont disponibles à partir de la
                    logithèque en tenant compte du statut du demandeur à l'Université. »Source : 
        \href{
        http://www.dgtic.umontreal.ca/logiciels/
        } {
        http://www.dgtic.umontreal.ca/logiciels/
        }
    .
            
            Certains logiciels de la logithèque de l'Université de Montréal sont donc rendus
                disponibles aux étudiants et aux employés pour installation sur leur poste personnel
                (Microsoft Office, EndNote, Oxygen, QDA Miner, SPSS, VirusScan, etc.). Notez bien
                les règles d'utilisation de ces produits qui figurent sur le site de la logithèque
                :
            
                « 2. Le droit d'utilisation d'un logiciel cesse à l'avènement de l'un des
                    événements suivants :
                a. Le changement de statut de la personne ne lui permet plus de l'utiliser.
                b. L'UdeM met fin à l'entente de renouvellement du logiciel avec l'éditeur.
                Dans l'un ou l'autre de ces cas, la personne s'engage à désinstaller le logiciel
                    de son poste de travail. »
            
            
        \textbf{
        Accès à la suite Microsoft Office
        }
    
            Le personnel et les étudiants bénéficient gratuitement de licences Microsoft Office
                365 Pro Plus pour l’installation sur des appareils personnels.
            Pour plus de renseignements et pour savoir comment obtenir ce service, rendez-vous à
                la page 
        \href{
        https://wiki.umontreal.ca/display/SIE/Office+365+ProPlus
        } {
        https://wiki.umontreal.ca/display/SIE/Office+365+ProPlus
        }
    .
            
                Note : nous vous recommandons d’installer la même version que celle qui est
                    installée sur les postes des laboratoires de l’EBSI et de la FAS; pour l’année
                    universitaire 2016-2017, les laboratoires disposeront la version 2016 de la
                    suite.
            
            
        \textbf{
        Autres logiciels sous licence institutionnelle
        }
    
            Pour accéder à la logithèque (
        \href{
        http://logitheque.dgtic.umontreal.ca/
        } {
        http://logitheque.dgtic.umontreal.ca/
        }
    ),
                vous devez vous authentifier avec votre code d'accès et votre UNIP / mot de
                passe.
        
    
    
        \newpage
        \section {
        Bibliothèques (1er, 2e et 3e cycles)
        }
        
        
        
        \textit{
        Guides > Soutien à l'enseignement > Bibliothèques (1er, 2e et 3e cycles) -
            EBSI - Université de Montréal - Confluence
        }
    
        Bibliothèques (1er, 2e et 3e cycles)
        
            La Bibliothèque des lettres et sciences humaines (BLSH) est située au pavillon
                Samuel-Bronfman (3000, rue Jean-Brillant). Elle relève de la Direction des
                bibliothèques de l'Université de Montréal. Au 3e étage de la bibliothèque
                sont logées les collections de monographies et de périodiques imprimés spécialisées
                en bibliothéconomie et sciences de l'information (BSI). Pour connaître les horaires
                d'ouvertures, consultez le site des web des bibliothèques (
        \href{
        http://www.bib.umontreal.ca/horaires/
        } {
        http://www.bib.umontreal.ca/horaires/
        }
    ). Notez que les heures
                d'ouverture sont généralement étendues en période d'examen. En tout temps, vous
                pouvez contacter la bibliothécaire de référence pour la collection BSI.
            
                Site de la BSI : 
        \href{
        http://guides.bib.umontreal.ca/disciplines/166-Bibliotheconomie-et-sciences-de-l-information
        } {
        http://guides.bib.umontreal.ca/disciplines/166-Bibliotheconomie-et-sciences-de-l-information
        }
    
                Responsable : Aminata Keita, bureau 3017
        \href{
        mailto:aminata.keita@umontreal.ca
        } {
        aminata.keita@umontreal.ca
        }
    , 514 343-6111, poste 5411
            
            La réserve de cours, située au comptoir de prêt, permet aux étudiants d'emprunter les
                documents qui y ont été laissés par les professeurs pour la consultation ou la
                reproduction sur place (
        \href{
        http://www.bib.umontreal.ca/pret/reserve-cours.htm
        } {
        http://www.bib.umontreal.ca/pret/reserve-cours.htm
        }
    ). Pour avoir accès au
                prêt, il suffit de se présenter à la bibliothèque avec sa carte UdeM (
        \href{
        http://www.bib.umontreal.ca/pret/
        } {
        http://www.bib.umontreal.ca/pret/
        }
    ).
            Les livres empruntés dans le réseau des bibliothèques de l'Université de Montréal
                peuvent tous être retournés à la bibliothèque des lettres et sciences humaines. Les
                livres retournés en dehors des heures d'ouverture peuvent être déposés dans la chute
                à documents située à l'extérieur de l'entrée (sous l'escalier).
            Plusieurs autres services vous sont offerts, dont la formation aux ressources
                documentaires. Le calendrier des activités de formation est publié sur le site web
                des bibliothèques; il est possible de s'inscrire en ligne (
        \href{
        http://www.bib.umontreal.ca/GIF/formations.aspx
        } {
        http://www.bib.umontreal.ca/GIF/formations.aspx
        }
    ). Un ensemble
                d'autres services offerts sont par la bibliothèque (
        \href{
        http://www.bib.umontreal.ca/SS/etudiants.htm
        } {
        http://www.bib.umontreal.ca/SS/etudiants.htm
        }
    ).
        
    
    
        \newpage
        \section {
        Dépôt institutionnel Papyrus
        }
        
        
        
        \textit{
        Guides > Soutien à l'enseignement > Dépôt institutionnel Papyrus - EBSI -
            Université de Montréal - Confluence
        }
    
        Dépôt institutionnel Papyrus
        
            Papyrus est le nom donné au dépôt institutionnel numérique de l'Université de
                Montréal accessible à l'adresse 
        \href{
        http://papyrus.bib.umontreal.ca
        } {
        http://papyrus.bib.umontreal.ca
        }
    . Ce dépôt vise la conservation permanente
                des documents qui y sont déposés et permet la diffusion sur Internet de
                publications, communications et documents pédagogiques produits par les différentes
                communautés de l'Université de Montréal (professeurs, chercheurs, étudiants,
                etc.).
            L'EBSI a développé sa propre vitrine dans Papyrus (
        \href{
        https://papyrus.bib.umontreal.ca/xmlui/handle/1866/559
        } {
        https://papyrus.bib.umontreal.ca/xmlui/handle/1866/559
        }
    ) où peuvent être
                déposés des documents de recherche, en français et en anglais, des travaux
                d'étudiants et des documents pédagogiques en sciences de l'information rédigés dans
                le cadre des activités de l'École. Les professeurs, professionnels, chargés de cours
                et étudiants peuvent y déposer des documents. Les personnes intéressées peuvent
                communiquer avec Minh Thi Trinh (
        \href{
        mailto:minh.thi.trinh@umontreal.ca
        } {
        minh.thi.trinh@umontreal.ca
        }
    , 514
                343-6111, poste 1292) pour connaître les modalités de diffusion. Les directives pour
                le dépôt de documents par les membres de l'EBSI sont disponibles à l'adresse 
        \href{
        http://www.ebsi.umontreal.ca/recherche/depot-numerique/guide-deposant/
        } {
        http://www.ebsi.umontreal.ca/recherche/depot-numerique/guide-deposant/
        }
    .
        
    
    
        \newpage
        \section {
        Association étudiante (1er, 2e et 3e cycles)
        }
        
        
        
        \textit{
        Guides > Vie étudiante > Association étudiante (1er, 2e et 3e cycles) - EBSI -
            Université de Montréal - Confluence
        }
    
        Association étudiante (1er, 2e et 3e cycles)
        
            L'association étudiante de l'EBSI se nomme AEEEBSI (Association des étudiantes et des
                étudiants de l'École de bibliothéconomie et des sciences de l'information); elle
                représente les étudiants inscrits aux programmes de 1er, 2e et
                    3e cycles offerts par l'EBSI.
            L'AEEEBSI a des représentants auprès de certains comités de l'École ainsi qu'à
                l'Assemblée de département. Elle crée elle-même ses propres comités pour
                l'organisation de diverses activités notamment des publications
                étudiantes, comme le journal étudiant La Référence disponible en ligne
                    (
        \href{
        http://lareference.ebsi.umontreal.ca/
        } {
        http://lareference.ebsi.umontreal.ca/
        }
    ).
            L'association organise aussi des activités sociales. Elle tient deux assemblées
                générales ordinaires par an. L'une à la rentrée pour l'élection des représentants
                étudiants sur les différents comités de l'École et de l'AEEEBSI, l'autre en janvier
                pour l'élection de l'exécutif.
            L'AEEEBSI met à la disposition de ses membres un local portant le nom de Café Melvil
                (C-2056). On y trouve notamment un four à micro-ondes, une bouilloire, des tables et
                un téléphone de courtoisie pour appels locaux seulement (514 343-6111, poste 3069).
                Vous devez composer le « 9 » avant tout appel à l'extérieur.
            
                
                    
        \href{
        http://aeeebsi.ebsi.umontreal.ca
        } {
        asso.ebsi@gmail.comhttp://aeeebsi.ebsi.umontreal.ca
        }
    
                
            
        
    
    
        \newpage
        \section {
        Centre d'émission de la carte UdeM
        }
        
        
        
        \textit{
        Guides > Vie étudiante > Services universitaires aux étudiants > Centre
            d'émission de la carte UdeM - EBSI - Université de Montréal - Confluence
        }
    
        Centre d'émission de la carte UdeM
        
            L'Université de Montréal met à la disposition des étudiants ainsi que de tout son
                personnel salarié une carte d'identité avec photographie. Cette carte sert à
                l'identification officielle lorsque c'est requis (par exemple lors de
                    l'emprunt d'équipement à un point de services techniques ou pour s'identifier
                    lors des examens) et permet d'emprunter dans les bibliothèques de
                l'Université. Tous les membres de la communauté universitaire devraient posséder
                leur carte d'identité valide. Toute personne qui circule à l'intérieur des immeubles
                en dehors des heures normales d'ouverture doit obligatoirement avoir sa carte
                d'identité UdeM sur soi et pouvoir la produire si elle est demandée pour fins de
                vérification.
            Pour les détails au sujet des formalités à remplir pour
                obtenir la carte UdeM, ainsi que pour connaître les dates d’émission massive et
                l’emplacement des bornes d’appro­visionnement, consulter le site web du centre
                d’émission.
            Centre d’émission de la
                    carte UdeMPavillon J.‑A.‑DeSève,2332, boul.
                Édouard-Montpetit, rez-de-chausséeMétro Édouard-Montpetit
        \href{
        http://www.carte.umontreal.ca
        } {
        http://www.carte.umontreal.ca
        }
    
        
    
    
        \newpage
        \section {
        Service d'impression
        }
        
        
        
        \textit{
        Guides > Vie étudiante > Services universitaires aux étudiants > Service
            d'impression - EBSI - Université de Montréal - Confluence
        }
    
        Service d'impression
        
            Les étudiants peuvent imprimer les notes de cours sur l’imprimante réseau des
                laboratoires d’informatique ou sur les autres imprimantes en accès libre du campus
                au coût de 8 ¢/page* pour une impression recto en format lettre.
        
        
            
            
                *
                Le coût pourrait changer avec le nouveau système qui sera implanté d’ici la
                    rentrée.
            
        
    
    
        \newpage
        \section {
        Ombudsman (1er, 2e et 3e cycles)
        }
        
        
        
        \textit{
        Guides > Vie étudiante > Services universitaires aux étudiants > Ombudsman
            (1er, 2e et 3e cycles) - EBSI - Université de Montréal - Confluence
        }
    
        Ombudsman (1er, 2e et 3e cycles)
        
            Vous pouvez vous adresser à l'ombudsman de l'Université de Montréal pour une demande
                d'information, un conseil ou une intervention, si vous estimez être victime
                d'injustice ou de discrimination et que vous avez épuisé les recours à votre
                disposition.
            L'ombudsman fera enquête s'il le juge nécessaire, évaluera le bien-fondé de la
                demande et, s'il y a lieu, transmettra ses recommandations aux autorités
                compétentes. Totalement indépendant vis-à-vis de la direction de l'Université,
                l'ombudsman exerce sa fonction de façon impartiale et est tenu à la confidentialité.
                Son rôle est de s'assurer du traitement juste et équitable, par l'administration de
                l'Université, de chacun des membres de la communauté universitaire.
            
                Téléphone : 514 343-2100
        \href{
        http://www.ombuds.umontreal.ca/
        } {
        ombudsman@umontreal.cahttp://www.ombuds.umontreal.ca/
        }
    
            
        
    
    
        \newpage
        \section {
        Bureau d'intervention en matière de harcèlement (1er, 2e et 3e cycles)
        }
        
        
        
        \textit{
        Guides > Vie étudiante > Services universitaires aux étudiants > Bureau
            d'intervention en matière de harcèlement (1er, 2e et 3e cycles) - EBSI - Université
            de Montréal - Confluence
        }
    
        Bureau d'intervention en matière de harcèlement (1er, 2e et 3e cycles)
        
            Le Bureau d'intervention en matière de harcèlement (BIMH) de l'Université de Montréal
                est situé au 3535, Ch. Queen Mary (coin Côte-des-Neiges), bureau 207. Au service de
                l'ensemble de la communauté universitaire, le rôle du bureau est de prévenir et
                d'intervenir. Les coordonnées du BIMH sont les suivantes :
            
                Téléphone : 514 343-7020
        \href{
        http://www.harcelement.umontreal.ca/
        } {
        harcelement@umontreal.cahttp://www.harcelement.umontreal.ca/
        }
    
            
        
    
    
        \newpage
        \section {
        Services aux étudiants (SAE) (1er, 2e et 3e cycles)
        }
        
        
        
        \textit{
        Guides > Vie étudiante > Services universitaires aux étudiants > Services
            aux étudiants (SAE) (1er, 2e et 3e cycles) - EBSI - Université de Montréal -
            Confluence
        }
    
        Services aux étudiants (SAE) (1er, 2e et 3e cycles)
        
            Les Services aux étudiants de l'Université de Montréal sont regroupés au sein du
                Pavillon J. A. DeSève. On y retrouve notamment les services suivants :
            
        \begin{itemize}
        
                
        \item action humanitaire et communautaire
                
        \item activités culturelles
                
        \item bureau de l'aide financière
                
        \item bureau des bourses d'études
                
        \item bureau des étudiants handicapés
                
        \item bureau des étudiants internationaux
                
        \item bureau du logement hors campus
                
        \item consultation psychologique; santé
                
        \item soutien aux études et développement de carrière.
            
        \end{itemize}
    
            
                Services aux étudiantsPavillon J. A. DeSève2332, boul.
                    Édouard-MontpetitTéléphone : 514 343-PLUS (7587)
        \href{
        http://www.sae.umontreal.ca/
        } {
        info@sae.umontreal.cahttp://www.sae.umontreal.ca/
        }
    
            
        
    
    
        \newpage
        \section {
        Vestiaire (1er, 2e et 3e cycles)
        }
        
        
        
        \textit{
        Guides > Vie étudiante > Services universitaires aux étudiants > Vestiaire
            (1er, 2e et 3e cycles) - EBSI - Université de Montréal - Confluence
        }
    
        Vestiaire (1er, 2e et 3e cycles)
        
            Étant donné le grand achalandage dans les laboratoires d'informatique de l'École, il
                est recommandé aux étudiants de louer un vestiaire à coût modique, afin de pouvoir y
                déposer manteau, bottes, etc.
            S'adresser à la Régie des immeubles à l'entrée du pavillon 3200 Jean Brillant et
                présenter votre carte d'étudiant (
        \href{
        http://www.di.umontreal.ca/communaute/casier_velo.html
        } {
        http://www.di.umontreal.ca/communaute/casier_velo.html
        }
    ).
        
    
    
        \newpage
        \section {
        Conférences midi
        }
        
        
        
        \textit{
        Guides > Vie étudiante > Activités parascolaires à l'EBSI > Conférences
            midi - EBSI - Université de Montréal - Confluence
        }
    
        Conférences midi
        
            Le programme des conférences midi organisées par l'EBSI porte sur des thèmes
                d’intérêt général ou d’actualité reliés au domaine de l’information. Les
                conférenciers proviennent du milieu professionnel ou universitaire québécois,
                canadien ou international. Ces conférences se tiennent de 11 h 45 à 12 h 45 et sont
                ouvertes à tous. Surveillez les annonces sur les babillards et sur le site web de
                l'École. Apportez votre lunch.
        
    
    
        \newpage
        \section {
        Description (1er cycle arv)
        }
        
        
        
        \textit{
        Guides > Certificat en archivistique > Description (1er cycle arv) - EBSI -
            Université de Montréal - Confluence
        }
    
        Description (1er cycle arv)
        
            
        \textbf{
        Présentation
        }
    
            Le rôle de l'archiviste est de soutenir les organisations et la société dans la
                gestion, l'exploitation et la sauvegarde de leur information, y compris de celle à
                valeur patrimoniale. L'archiviste intervient tout au long du cycle de vie des
                archives depuis leur création jusqu'à leur élimination ou leur conservation
                permanente.
            L'archiviste
            
        \begin{itemize}
        
                
        \item conseille les administrations sur la gestion de l'information administrative
                    utilisée quotidiennement ou occasionnellement;
                
        \item conçoit des systèmes manuels ou informatisés de gestion des archives courantes,
                    intermédiaires et définitives;
                
        \item supervise ou exécute des tâches relatives à la création, à l'évaluation, à
                    l'acquisition, à la classification, à la description, à l'indexation, à la
                    préservation et à la diffusion des archives.
            
        \end{itemize}
    
            L'archiviste travaille dans des milieux aussi divers que les entreprises privées, les
                organismes gouvernementaux, le milieu de l'éducation, le réseau municipal et celui
                des organismes de santé et de services sociaux.
            
        \textbf{
        Perspectives d'emploi
        }
    
            Les administrateurs sont maintenant sensibilisés à la gestion des archives. Ils sont
                confrontés à une masse d'information toujours grandissante, à des obligations créées
                par la législation et à l'impact des technologies de l'information. Ils engagent des
                spécialistes en archivistique pour gérer l'ensemble de leurs archives ou pour régler
                des problèmes ponctuels.
            
        \textbf{
        Buts du programme
        }
    
            Le programme de certificat en archivistique a pour but de former des praticiens
                appelés à œuvrer dans le domaine des archives en y appliquant l'approche intégrée de
                l'archivistique pour répondre aux besoins de l'administration et de la recherche,
                tout en y cultivant un esprit de service à la collectivité.
            Le cursus proposé vise ainsi à former des diplômés qui auront acquis des
                connaissances de base théoriques et pratiques en archivistique, qui auront développé
                des habiletés pour concevoir, appliquer et utiliser des outils archivistiques et qui
                auront développé des aptitudes propres à l'exercice des fonctions
                archivistiques.
            
        \textbf{
        Objectifs du programme
        }
    
            À la fin de ses études, tout étudiant doit être en mesure :
            
        \begin{itemize}
        
                
        \item d'appliquer les principes et les méthodes archivistiques;
                
        \item de comprendre les principes de gestion des services d'archives;
                
        \item de concevoir et d'appliquer des systèmes de création, de traitement, de
                    conservation et d'utilisation des archives peu importe leur forme et leur
                    support (archives informatiques, iconographiques, cartographiques et
                    audiovisuelles);
                
        \item de participer à l'évaluation des archives;
                
        \item de collaborer au choix et à la mise en œuvre des systèmes informatisés de
                    gestion des archives;
                
        \item de comprendre le milieu archivistique québécois, canadien et international;
                
        \item d'intégrer les connaissances, les habiletés et les attitudes acquises tout au
                    long de sa formation.
            
        \end{itemize}
    
        
    
    
        \newpage
        \section {
        Structure du programme (1er cycle arv)
        }
        
        
        
        \textit{
        Guides > Certificat en archivistique > Structure du programme (1er cycle arv) -
            EBSI - Université de Montréal - Confluence
        }
    
        Structure du programme (1er cycle arv)
        
            Numéro du programme : 1-056-5-1
            
                
                    10 cours — 30 crédits
                
            
            
        \begin{itemize}
        
                
        \item 7 cours OBLIGATOIRES — 21 crédits : Bloc 70A
                
        \item 2 cours À OPTION — 6 crédits : Bloc 70B
                
        \item 1 cours AU CHOIX — 3 crédits : Bloc 70Z
            
        \end{itemize}
    
            
                
                    
                        
                            
                                Bloc 70A : obligatoire (21 crédits)
                            
                        
                        
                            
                                ARV1050 Introduction à l'archivistique
                            
                            
                                Concomitant à tous les cours de sigle ARV à
                                        l'exception du ARV2955; doit être suivi dès
                                        le premier trimestre d'inscription
                            
                        
                        
                            
                                ARV1052 Typologie des archives
                            
                            
                                Concomitant : ARV1050
                            
                        
                        
                            
                                ARV1053 Évaluation des archives
                            
                            
                                Concomitant : ARV1050
                            
                        
                        
                            
                                ARV1054 Classification des archives
                            
                            
                                Concomitant : ARV1050
                            
                        
                        
                            
                                ARV1055 Description des archives
                            
                            
                                Concomitant : ARV1050
                            
                        
                        
                            
                                ARV1058 Administration des archives
                            
                            
                                Concomitant : ARV1050
                            
                        
                        
                            
                                ARV3054 Gestion des archives numériques
                            
                            
                                Concomitant : ARV1050 ou équivalent
                            
                        
                        
                            
                                Bloc 70B : à option (6 crédits)
                            
                        
                        
                            
                                ARV1056 Diffusion, communication et exploitation
                            
                            
                                Concomitant : ARV1050
                            
                        
                        
                            
                                ARV1057 Stage
                            
                            
                                
                                    Pré-requis :
                                
                                
        \begin{itemize}
        
                                    
        \item 24 crédits de sigle ARV et/ou INU dans le programme
                                    
        \item Approbation requise de la coordonnatrice de stages (moyenne
                                        cumulative de 3,0/4,3 et qualité du français (réussite du
                                        test de français d'admission si applicable)
                                
        \end{itemize}
    
                            
                        
                        
                            
                                ARV1060 Archives et société
                            
                            
                                Concomitant : ARV1050
                            
                        
                        
                            
                                ARV1061 Archives non textuelles
                            
                            
                                Concomitant : ARV1050
                            
                        
                        
                            
                                ARV2955 Histoire du livre et de l'imprimé
                            
                             
                        
                        
                            
                                ARV3051 Préservation des archives
                            
                            
                                Concomitant : ARV1050 ou équivalent
                            
                        
                        
                            
                                ARV3052 Activités dirigées
                            
                            
                                Pré-requis : 15 crédits ARV complété dans le
                                    programme
                            
                        
                        
                            
                                ARV3053 Archives et info. : aspects juridiques
                            
                            
                                Concomitant : ARV1050 ou équivalent
                            
                        
                        
                            
                                INU1001 Introduction à l'information numérique
                            
                            
                                Préalable ou concomitant à tous les cours de sigle INU
                            
                        
                        
                            
                                INU1010 Création de l'information numérique
                            
                            
                                Concomitant : INU1001
                            
                        
                        
                            
                                INU1030 Préservation de l'information numérique
                            
                            
                                Concomitant : INU1001
                            
                        
                        
                            
                                INU1050 Diffusion d'information numérique
                            
                            
                                Concomitant : INU1001
                            
                        
                        
                            
                                Bloc 70Z : cours au choix (3 crédits)
                            
                        
                        
                            
                                Un cours choisi parmi les cours à option du certificat en
                                    archivistique ou choisi dans un autre programme de l'Université
                                    de Montréal parmi les cours ouverts comme cours au choix. Pour
                                    facilement repérer les cours ouverts comme cours au choix,
                                    cochez l'option « Cours au choix » dans les particularités de la
                                    recherche effectuée sur le site 
                                
        \href{
        http://admission.umontreal.ca/cours-de-1er-cycle/
        } {
        http://admission.umontreal.ca/cours-de-1er-cycle/
        }
    
                                .
                            
                        
                    
                
            
        
    
    
        \newpage
        \section {
        Cours du programme (1er cycle arv)
        }
        
        
        
        \textit{
        Guides > Certificat en archivistique > Cours du programme (1er cycle arv) -
            EBSI - Université de Montréal - Confluence
        }
    
        Cours du programme (1er cycle arv)
        
            Une description détaillée des cours (objectifs, évaluation, calendrier) est
                disponible dans les plans de cours sur le site web de l'EBSI (
        \href{
        http://cours.ebsi.umontreal.ca/
        } {
        http://cours.ebsi.umontreal.ca/
        }
    ). La description des cours de sigle INU se
                retrouve à la section 7.6. Ci-dessous se retrouvent les descriptions courtes des
                cours de sigles ARV.
            
        \textbf{
        ARV1050 — Introduction à l'archivistique, 3 crédits (cours obligatoire)
        }
    
            Méthode de travail. Ressources documentaires. Histoire. Disciplines apparentées.
                Terminologie. Catégories et types de documents. Notions fondamentales. Fonctions.
                Législation. Institutions et réseaux. Présentation d'un logiciel de gestion.
            
                Concomitant à tous les cours de sigle ARV à l'exception du cours ARV2955.
            
            
        \textbf{
        ARV1052 — Typologie des archives, 3 crédits (cours obligatoire)
        }
    
            Catégories, types de documents et particularités. Assises légales ou administratives.
                Modes de création. Étude de corpus spécifiques. Production informatique de documents
                analogiques et numériques.
            
                Concomitant : ARV1050 Introduction à l'archivistique
            
            
        \textbf{
        ARV1053 — Évaluation des archives, 3 crédits (cours obligatoire)
        }
    
            Valeurs des archives. Processus d'évaluation et critères d'évaluation. Application à
                la gestion du cycle de vie : calendrier de conservation, acquisition et versement,
                élimination. Application informatique spécifique aux documents analogiques et
                numériques.
            
                Concomitant : ARV1050 Introduction à l'archivistique
            
            
        \textbf{
        ARV1054 — Classification des archives, 3 crédits (cours obligatoire)
        }
    
            Classification et autres métadonnées. Méthodes et modes de classement. Contexte du
                respect des fonds. Création d'outils de repérage : index, contrôle d'autorité.
                Application informatique spécifique aux documents analogiques et numériques.
            
                Concomitant : ARV1050 Introduction à l'archivistique
            
            
        \textbf{
        ARV1055 — Description des archives, 3 crédits (cours obligatoire)
        }
    
            Évolution des pratiques de description des archives. Règles nationales et
                internationales de description. Application des règles nationales de description.
                Application informatique spécifique aux documents analogiques et numériques.
            
                Concomitant : ARV1050 Introduction à l'archivistique
            
            
        \textbf{
        ARV1056 — Diffusion, communication et exploitation, 3 crédits (cours à option)
        }
    
            Méthodologie des études de clientèles. Mise en valeur des archives. Exploitation des
                outils de repérage. Conception d'outils de diffusion. Éthique et déontologie.
                Législation. Archives en environnement réseauté. Base de données du réseau.
            
                Concomitant : ARV1050 Introduction à l'archivistique
            
            
        \textbf{
        ARV1057 — Stage, 3 crédits (cours à option)
        }
    
            D'une durée de 25 jours ouvrables réalisés dans un milieu de travail, le stage est
                accompagné d'activités pédagogiques à l'Université. Possibilité d'exemption du cours
                pour expérience de travail pertinente en archivistique. Approbation requise.
            
                Préalable : 24 crédits de sigle ARV et/ou INU dans le programme.
            
            Approbation requise : moyenne de 3,0 sur 4,3 pour un minimum de
                quatre cours du programme ; qualité du français écrit et parlé (test d'admission
                réussi si applicable).
            Le stage s'intègre dans le certificat en archivistique comme un complément de
                formation essentiel à l'étudiant qui se prépare à travailler dans le milieu
                archivistique. Tout en s'initiant à la réalité quotidienne l'étudiant applique les
                principes, méthodes et processus acquis, et réfléchit sur les différences
                d'application. Le stage lui permet donc d'analyser et de synthétiser les éléments de
                la pratique d'un milieu pour ensuite se situer professionnellement.
            Le stage se réalise aux trimestres d'automne, d'hiver et d'été. Il est à noter qu'il
                est impossible de terminer le certificat en archivistique en deux trimestres pour un
                étudiant qui fait le stage. D'une durée de 25 jours ouvrables, il est réalisé dans
                un milieu identifié comme milieu d'intervention, à un rythme minimum de trois jours
                consécutifs par semaine.
            Le stage permet à l'étudiant :
            
        \begin{itemize}
        
                
        \item d'intégrer les connaissances théoriques à une expérience pratique;
                
        \item de connaître un organisme, sa politique et son programme archivistique;
                
        \item de développer des habiletés professionnelles;
                
        \item de développer des qualités personnelles;
                
        \item de vivre une expérience de la profession;
                
        \item de se situer professionnellement;
                
        \item d'avoir un aperçu des différents milieux de travail et des programmes
                    archivistiques.
            
        \end{itemize}
    
            
                ÉVALUATION
            
            
        \begin{itemize}
        
                
        \item Par le responsable du milieu hôte : 40 %
                
        \item Par la coordonnatrice des stages : 60 % (rencontres pendant le stage, journal de
                    bord, rapport de stage)
            
        \end{itemize}
    
            
                ORGANISATION DU STAGE
            
            Une rencontre individuelle de l'étudiant avec la coordonnatrice des stages pour la
                planification du stage : cette rencontre doit se dérouler avant la semaine de
                lecture du trimestre précédant le stage. Le « Formulaire de préinscription pour le
                stage » doit alors être rempli et accompagné du curriculum vitæ pour cette
                rencontre.
            Deux rencontres de groupe :
            
        \begin{itemize}
        
                
        \item une rencontre avant le stage, à la suite de l'inscription;
                
        \item une rencontre pendant le stage.
            
        \end{itemize}
    
            
        \textbf{
        ARV1058 — Administration des archives, 3 crédits (cours obligatoire)
        }
    
            Organisations, processus d'affaires et documents. Éléments de gestion d'un service
                d'archives et de ses ressources. Politiques et procédures de gestion des archives.
                Gestion des documents semi-actifs. Analyse des besoins.
            
                Concomitant : ARV1050 Introduction à l'archivistique
            
            
        \textbf{
        ARV1060 — Archives et société, 3 crédits (cours à option) [cours non offert en
                2016–2017]
        }
    
            Archives et mémoire collective. Place des archives dans le domaine de l'histoire et
                du patrimoine. Société consommatrice des archives. Place et rôle des archives et des
                archivistes dans la société. Archives dans la société de l'information.
            
                Concomitant : ARV1050 Introduction à l'archivistique
            
            
        \textbf{
        ARV1061 — Archives non textuelles, 3 crédits (cours à option)
        }
    
            Typologie des archives non textuelles. Identification des supports et formats.
                Traitement propre aux archives non textuelles. Classification. Indexation.
                Description. Problématique de stockage. Potentiel de diffusion des archives non
                textuelles.
            
                Concomitant : ARV1050 Introduction à l'archivistique
            
            
        \textbf{
        ARV2955 — Histoire du livre et de l'imprimé, 3 crédits (cours à option)
        }
    
            Les trois révolutions du livre, de la naissance de l'imprimé à l'arrivée de
                l'informatique; l'histoire du livre dans le contexte québécois et canadien. Ensemble
                du circuit de l'imprimé, de la production d'un manuscrit à sa distribution.
            
        \textbf{
        ARV3051 — Préservation des archives, 3 crédits (cours à option)
        }
    
            Nature, finalité et planification stratégique de la préservation. Contexte
                multimédia. Normes de protection. Prévention, conservation et restauration. Modes de
                protection des fichiers informatiques : repiquage, migration et émulation.
            
                Concomitant : ARV1050 Introduction à l'archivistique ou équivalent
            
            
        \textbf{
        ARV3052 — Activités dirigées, 3 crédits (cours à option)
        }
    
            Étude de cas ou recherche dirigée. Réservé aux étudiants détenant ou terminant un
                baccalauréat, à l'exclusion des étudiants de maîtrise en sciences de l'information.
                Ce cours de synthèse consiste en un travail individuel de l'étudiant sous
                supervision d'un responsable. Il prend la forme d'un travail de recherche, d'une
                étude de cas réalisée dans un milieu de travail ou d'un deuxième stage (le ARV1057
                devient alors un préalable; pour les modalités pédagogiques, se référer à la
                description du ARV1057).
            
                Préalable : 15 crédits ARV.
            
            
                OBJECTIFS GÉNÉRAUX
            
            Ce cours permettra à l'étudiant de :
            
        \begin{itemize}
        
                
        \item produire une réflexion personnelle après en avoir planifié les différentes
                    étapes.
            
        \end{itemize}
    
            
                OBJECTIFS SPÉCIFIQUES
            
            Au terme du cours, l'étudiant sera en mesure de :
            
        \begin{itemize}
        
                
        \item établir une synthèse des connaissances, des habiletés et des aptitudes acquises
                    dans les cours du certificat;
                
        \item développer ses habiletés à planifier et à mener une recherche de façon autonome
                    en élaborant un plan de recherche et en le réalisant;
                
        \item rédiger des synthèses de la documentation en présentant les principaux courants
                    de pensée relatifs à son sujet et en identifiant les principaux auteurs;
                
        \item recueillir les données pertinentes à sa recherche par des méthodes
                    appropriées;
                
        \item analyser les données recueillies en en présentant les résultats selon les usages
                    en vigueur.
            
        \end{itemize}
    
            
                PROCÉDURES
            
            Avant de s'inscrire à ce cours, l'étudiant devra à l'aide du formulaire disponible à
                la fin de la présente section :
            
        \begin{itemize}
        
                
        \item Établir les contacts avec la personne qui le supervisera et obtenir son accord
                    écrit sur :la supervision de l'étudiant et la supervision du projet
                            présenté par l'étudiant.
                
        \item Présenter un projet en un minimum de deux pages comportant :le titre du
                            projet : énoncé du sujet du travail;les objectifs du projet :
                            questions orientant la recherche ou hypothèses à vérifier;la
                            méthodologie privilégiée pour le travail;la description du
                            projet : les grandes lignes du contenu;le calendrier de travail
                            : les étapes de réalisation du projet;l'apport du projet à sa
                            formation en archivistique.
                
        \item Faire approuver son projet par le responsable du certificat en
                    archivistique.
            
        \end{itemize}
    
            
                
        \href{
        /download/attachments/124097591/arv3052_formulaire.pdf?version=1&modificationDate=1465306272000&api=v2
        } {
        FORMULAIRE POUR INSCRIPTION
        }
    
            
            
        \textbf{
        ARV3053 — Archives et info. : aspects juridiques, 3 crédits (cours à option)
        }
    
            Application des lois et règlements dans un contexte archivistique québécois et
                canadien. Jurisprudence. Accès à l'information. Renseignements personnels.
                Technologies de l'information. Droit d'auteur. Dépôt légal. Niveaux d'application.
                Déontologie.
            
                Concomitant : ARV1050 Introduction à l'archivistique ou équivalent
            
            
        \textbf{
        ARV3054 — Gestion des archives numériques, 3 crédits (cours obligatoire)
        }
    
            Caractéristiques des archives numériques. Fonctions et outils archivistiques dans
                l'environnement numérique. Analyse et implantation. Numérisation. Logiciels de
                gestion électronique des documents. Préservation. Sécurité.
            
                Concomitant : ARV1050 Introduction à l'archivistique ou équivalent
            
        
    
    
        \newpage
        \section {
        Description (1er cycle gin)
        }
        
        
        
        \textit{
        Guides > Certificat en gestion de l'information numérique > Description
            (1er cycle gin) - EBSI - Université de Montréal - Confluence
        }
    
        Description (1er cycle gin)
        
            
        \textbf{
        Présentation
        }
    
            Le travail du gestionnaire de l'information numérique consiste, typiquement, en l'une
                ou l'autre ou même plusieurs des fonctions suivantes en regard de l'information
                numérique :
            
        \begin{itemize}
        
                
        \item aide à sa création;
                
        \item aide à son acquisition;
                
        \item aide à son organisation (structuration, indexation, classification);
                
        \item aide à son stockage et à sa préservation;
                
        \item aide à sa recherche;
                
        \item aide à sa diffusion.
            
        \end{itemize}
    
            
        \textbf{
        Perspectives d'emploi
        }
    
            De plus en plus, les administrateurs sont sensibilisés à l'importance, aux enjeux et
                aux difficultés de bien gérer leur information numérique. Ils sont confrontés à une
                masse d'information numérique toujours grandissante, à des obligations créées par la
                législation et à l'impact des technologies de l'information. Ils sont donc de plus
                en plus convaincus de la nécessité d'embaucher des spécialistes pour gérer cette
                information.
            
        \textbf{
        Buts du programme
        }
    
            Le programme de certificat en gestion de l'information numérique a pour but de former
                des praticiens appelés à exercer des fonctions de gestion de l'information pouvant
                être lue et traitée par ordinateur.
            Le cursus proposé vise ainsi à former des diplômés qui auront acquis des
                connaissances de base théoriques et pratiques en gestion de l'information numérique,
                c'est-à-dire la création, l'organisation, la préservation, la recherche et la
                diffusion de cette information.
            
        \textbf{
        Objectifs du programme
        }
    
            L'étudiant qui aura complété le certificat en gestion de l'information numérique sera
                en mesure d'évoluer avec aisance dans un environnement de travail hautement
                technologique. Il pourra :
            
        \begin{itemize}
        
                
        \item reconnaître la place de l'information numérique dans la société;
                
        \item décrire les caractéristiques de l'information numérique;
                
        \item utiliser des logiciels de création d'information numérique ou bureautique;
                
        \item appliquer les méthodes d'organisation de l'information numérique;
                
        \item participer aux opérations de préservation de l'information numérique;
                
        \item comprendre les problèmes liés à l'évaluation de l'information numérique;
                
        \item appliquer les principes et les méthodes de recherche d'information
                    numérique;
                
        \item connaître les principales sources et ressources d'information numérique;
                
        \item appliquer les techniques de diffusion de l'information numérique.
            
        \end{itemize}
    
        
    
    
        \newpage
        \section {
        Structure du programme (1er cycle gin)
        }
        
        
        
        \textit{
        Guides > Certificat en gestion de l'information numérique > Structure du
            programme (1er cycle gin) - EBSI - Université de Montréal - Confluence
        }
    
        Structure du programme (1er cycle gin)
        
            Numéro du programme : 1-053-5-1
            
                
                    10 cours‡ — 30 crédits
                
            
            
        \begin{itemize}
        
                
        \item 6 cours OBLIGATOIRES — 18 crédits : Bloc 71A
                
        \item 3 cours À OPTION — 9 crédits : Bloc 71B
                
        \item 1 cours AU CHOIX — 3 crédits : Bloc 71z
            
        \end{itemize}
    
            
                
                    
                        
                            
                                Bloc 71A : obligatoire (18 crédits)
                            
                        
                        
                            INU1001 Introduction à l'information numérique
                            Cours préalable ou concomitant à tous les cours
                                de sigle INU (doit être suivi dès le premier trimestre
                                d'inscription)
                        
                        
                            INU1010 Création de l'information numérique
                            Concomitant : INU1001
                        
                        
                            INU1020 Organisation de l'information numérique
                            Concomitant : INU1001
                        
                        
                            
                                INU1030 Préservation de
                                    l'information numérique
                            
                            Concomitant : INU1001
                        
                        
                            INU1040 Recherche de l'information numérique
                            Concomitant : INU1001
                        
                        
                            INU1050 Diffusion d'information numérique
                            Concomitant : INU1001 (à faire
                                idéalement à la fin du certificat)
                        
                        
                            
                                Bloc 71B : à option (9 crédits)
                            
                        
                        
                            INU3000 Projet d'intégration
                            IFT1941 Traitement et montage vidéo
                        
                        
                            
                                INU3011 Documents
                                    structurés
                                
                                    (Préalable : INU1001 ou équivalent)
                                
                            
                            
                                IFT2720 Introduction au
                                    multimédia
                            
                        
                        
                            
                                INU3051 Information et sites Web
                                
                                    (Préalable : INU1001 ou équivalent)
                                
                            
                            COM1500 Communication organisationnelle
                        
                        
                            
                                INU3052 Gestion de systèmes d'info. en réseau
                                
                                    (Préalable : INU1001 ou équivalent)
                                
                            
                            
                                COM1560 Communication infographique
                            
                        
                        
                            ARV1050 Introduction à l'archivistique
                            COM2021 Technologies et travail en collaboration
                        
                        
                            
                                ARV1052 Typologie des archives
                                
                                    (Concomitant : ARV1050)
                                
                            
                            COM2210 Impact des nouvelles technologies
                        
                        
                            ARV2955 Histoire du livre et de l'imprimé
                            COM2541 Titre à venir
                        
                        
                            
                                ARV3054 Gestion des archives numériques
                                
                                    (Concomitant : ARV1050 ou équivalent)
                                
                            
                            COM2571 Interfaces et scénarisation
                        
                        
                            IFT1810 Introduction à la programmation
                            
                                COM2590 Communautés virtuelles, réseaux sociaux
                            
                        
                        
                            IFT1912 Initiation aux chiffriers électroniques
                            COM2710 Innovation et médiation sociotechnique
                        
                        
                            IFT1931 Initiation aux bases de données 
                             
                        
                        
                            
                                Cours au choix (3 crédits)
                            
                        
                        
                            Un cours choisi parmi les cours à
                                option du certificat en gestion de l'information numérique ou choisi
                                dans un autre programme de l'Université de Montréal parmi les cours
                                ouverts comme cours au choix. Pour facilement repérer les cours
                                ouverts comme cours au choix, cochez l'option « Cours au choix »
                                dans les particularités de la recherche effectuée sur le site 
        \href{
        http://admission.umontreal.ca/cours-de-1er-cycle/
        } {
        http://admission.umontreal.ca/cours-de-1er-cycle/
        }
    . 
                        
                    
                
            
            ‡  Les cours INU sont contingentés à 80 étudiants par cours. À l'exception du
                cours INU1001, chaque cours INU comprend une partie théorique et une partie
                pratique.
        
    
    
        \newpage
        \section {
        Cours du programme (1er cycle gin)
        }
        
        
        
        \textit{
        Guides > Certificat en gestion de l'information numérique > Cours du
            programme (1er cycle gin) - EBSI - Université de Montréal - Confluence
        }
    
        Cours du programme (1er cycle gin)
        
            Une description détaillée des cours (objectifs, évaluation, calendrier) est
                disponible dans les plans de cours sur le site de l'EBSI à l'URL 
        \href{
        http://cours.ebsi.umontreal.ca/
        } {
        http://cours.ebsi.umontreal.ca/
        }
    . La description des cours de sigle ARV se
                retrouve à la rubrique 
        \href{
        /pages/viewpage.action?pageId=124097591
        } {
        Certificat
                    en archivistique > Cours du programme
        }
    . Pour plus d'information sur les
                cours de sigle IFT présents dans la liste des cours à option du programme, vous
                pouvez consulter le répertoire des cours de sigle IFT (
        \href{
        https://admission.umontreal.ca/repertoire-des-cours/filtres/cycle_1/matiere_IFT/
        } {
        http://admission.umontreal.ca/repertoire-des-cours/filtres/cycle_1/matiere_IFT/
        }
    ).
                Pour les cours de sigle COM, vous pouvez consulter le répertoire des cours de signe
                COM (
        \href{
        http://admission.umontreal.ca/repertoire-des-cours/filtres/cycle_1/matiere_com/
        } {
        http://admission.umontreal.ca/repertoire-des-cours/filtres/cycle_1/matiere_com/
        }
    )
                ou les plans de cours cadre sur le site du Département de communication (
        \href{
        http://com.umontreal.ca/fileadmin/Documents/FAS/Communication/Documents/1-Programmes-cours/Plans/Plans-de-cours-cadre_19mai2015.pdf
        } {
        http://com.umontreal.ca/fileadmin/Documents/FAS/Communication/Documents/1-Programmes-cours/Plans/Plans-de-cours-cadre_19mai2015.pdf
        }
    ).
                Ci-dessous se retrouvent les descriptions courtes des cours de sigles INU.
            
        \textbf{
        Cours obligatoires
        }
    
            INU1001 — Introduction à l'information numérique, 3 crédits (cours
                obligatoire)
            
                Préalable ou concomitant à tous les autres cours de sigle INU.
            
            Caractéristiques de l'information numérique. Aspects sociaux, économiques, éthiques
                et juridiques. Documents et fichiers numériques. Internet. Réseaux. Introduction à
                la création de sites web. Sécurité informatique. Compétences informationnelles.
            INU1010 — Création de l'information numérique, 3 crédits (cours
                obligatoire)
            
                Concomitant : INU1001 Introduction à l'information numérique
            
            Contextes et outils de création de l'information numérique. Typologies et formats des
                documents numériques. Bureautique. Modélisation. Bases de données. Images, fichiers
                sonores et vidéos. Numérisation et reconnaissance optique de caractères.
            INU1020 — Organisation de l'information numérique, 3 crédits (cours
                obligatoire)
            
                Concomitant : INU1001 Introduction à l'information numérique
            
            Classification, description et indexation de l'information sur poste de travail, en
                GED et sur le Web. Outils de navigation. Normes. Identifiants. Métadonnées.
                Vocabulaires contrôlés. Indexation automatique. Données ouvertes. Web
                sémantique.
            INU1030 — Préservation de l'information numérique, 3 crédits (cours
                obligatoire)
            
                Concomitant : INU1001 Introduction à l'information numérique
            
            Conservation et préservation des supports, données et contenus numériques. Enjeux et
                problématiques. Aspects juridiques. Projets et normes. Stratégies de sauvegarde et
                plan de secours. Formats. Intégrité et authenticité. Valorisation. Tendances.
            INU1040 — Recherche de l'information numérique, 3 crédits (cours
                obligatoire)
            
                Concomitant : INU1001 Introduction à l'information numérique
            
            Recherche d'information numérique (Internet, Web 2.0, catalogues, bases de données
                commerciales). Stratégies de recherche (logique booléenne, opérateurs, troncatures).
                Agents intelligents, alertes. Évaluation des résultats. Logiciel
                bibliographique.
            INU1050 — Diffusion d'information numérique, 3 crédits (cours obligatoire)
            
                Concomitant : INU1001 Introduction à l'information numérique
            
            Architecture de l'info. Comportement des utilisateurs. Utilisabilité, ergonomie. Web
                2.0. Réseaux sociaux. Systèmes de gestion de contenu, outils collaboratifs (blogue,
                Wiki, forum). Listes de distribution, syndication de contenu, baladodiffusion.
            
        \textbf{
        Cours à option
        }
    
            Veuillez noter que les cours sont offerts selon les disponibilités budgétaires.
            INU3000 — Projet d'intégration [non offert en 2016–2017], 3 crédits (cours à
                option)
            Étude de cas, recherche dirigée ou projet d'intervention supervisé dans une
                organisation. Réservé aux étudiants détenant ou terminant un baccalauréat (à
                l'exclusion des étudiants de maîtrise en sciences de l'information).
            INU3011 — Documents structurés, 3 crédits (cours à option)
            
                Préalable : INU1001 Introduction à l'information numérique ou équivalent
            
            Formats de documents et langages de balisage. Historique. Concepts de base. XML et
                normes périphériques. Modélisation, validation, stylage. Chaînes de traitement et
                méthodologies d'implantation. Bases de données XML.
            INU3051 — Information et sites Web, 3 crédits (cours à option)
            
                Préalable : INU1001 Introduction à l'information numérique ou équivalent
            
            Gestion de projets Web. Normes et standards du Web. Langage HTML et feuilles de style
                CSS. Design Web. Ergonomie. Accessibilité. Interactivité. Référencement et mesure
                d'audience.
            INU3052 — Gestion de systèmes d'information en réseau, 3 crédits (cours à
                option)
            
                Préalable : INU1001 Introduction à l'information numérique ou équivalent
            
             Implantation et gestion d'un serveur de fichiers, d'un serveur Web et d'un
                système de gestion de contenu. Sécurité informatique : technologies, chiffrement,
                gestion des droits, formation des utilisateurs, plan d'urgence. Virtualisation.
        
    
    
        \newpage
        \section {
        Programmes de certificat > Planification de votre programme (1er cycle)
        }
        
        
        
        \textit{
        Guides > Programmes de certificat > Planification de votre programme (1er
            cycle) - EBSI - Université de Montréal - Confluence
        }
    
        Programmes de certificat > Planification de votre programme (1er cycle)
        
            Avant de procéder à l'inscription par le Centre étudiant, vous devez planifier
                l'ensemble de votre programme puis de votre trimestre.
            
        \textbf{
        Rythme de vos études
        }
    
            Trimestre à plein temps : minimum de quatre cours (12 crédits).
            Le certificat en gestion de l'information numérique peut être complété en deux
                trimestres à plein temps (5 cours par trimestre) peu importe si vous êtes admis à
                l'automne ou à l'hiver. Il en va de même pour le certificat en archivistique sauf si
                vous désirez faire le stage. Il est en effet impossible de terminer le certificat en
                archivistique en deux trimestres si vous êtes admis à l'automne ou à l'hiver et que
                vous désirez faire le stage.
            Trimestre à temps partiel : un à trois cours (3 à 9 crédits)
                par trimestre maximum (a un impact sur les prêts et bourses).
            Scolarité : la scolarité maximale d'un programme de certificat est
                de quatre années civiles à compter de la première inscription.
            Inscription : Un étudiant qui n'a pas suivi de cours pendant trois
                trimestres consécutifs et qui souhaite s'inscrire au 4e trimestre suivant
                doit remplir une nouvelle demande d'admission, car son dossier devient inactif.
            
        \textbf{
        Choix des cours à chacun des trimestres
        }
    
            
        \begin{itemize}
        
                
        \item Tenez compte des cours obligatoires, des cours optionnels et vérifiez les
                    préalables.
                
        \item Respectez la structure de votre programme, c'est-à-dire complétez les cours
                    requis pour chacun des blocs de cours. Si cela n'est pas fait, vous ne pourrez
                    obtenir votre diplôme de certificat.
                
        \item Remarque : L'ordre numérique des cours n'est pas significatif quant à
                    la gradation de l'apprentissage, à l'exception du cours d'introduction qui doit
                    commencer le programme.
            
        \end{itemize}
    
            
        \textbf{
        Cours au choix et hors-programme
        }
    
            Le cours au choix (CH) peut être choisi parmi les cours à option de votre certificat
                ou dans un autre programme de l'Université de Montréal parmi les cours qui sont
                ouverts comme cours au choix.
            Si vous le prenez dans un autre département :
            
        \begin{itemize}
        
                
        \item Vérifiez si le cours se donne au trimestre choisi;
                
        \item Vérifiez l'horaire du cours choisi (le jour ou le soir);
                
        \item Vérifiez si le cours est ouvert à des étudiants d'autres programmes.
                
        \item Pour vous aider, consultez les horaires sur le Centre étudiant, à la rubrique
                    Recherche multiple à l'adresse : http://www.etudes.umontreal.ca/horaire/index.html.
            
        \end{itemize}
    
            Vous pouvez aussi suivre un ou deux cours hors-programme (HP) en plus des dix cours
                de votre certificat. Il ne faut ainsi pas confondre le « cours au choix », qui fait
                partie des dix cours à suivre dans le cadre de votre certificat, et des cours «
                hors-programme » qui eux, ne font pas partie des dix cours à suivre. Les notes
                obtenues pour les cours hors-programme ne sont pas comptabilisées dans la moyenne de
                votre certificat.
            
        \textbf{
        Équivalence de cours
        }
    
            Dans certains cas, il est possible d'accorder une équivalence pour des cours
                complétés à l'extérieur du certificat :
            
        \begin{itemize}
        
                
        \item Aucune équivalence ne peut être accordée sur la base d'un cours réussi au niveau
                    collégial.
                
        \item Aucune équivalence ne peut être accordée sur la base d'un cours réussi au sein
                    d'un certificat complété, mais non cumulé dans un baccalauréat.
                
        \item Le contenu, les objectifs, le niveau et le nombre de crédits du cours doivent
                    correspondre à un cours inscrit au programme.
                
        \item Une équivalence peut être accordée pour le cours au choix du programme lorsqu'un
                    étudiant a déjà obtenu un diplôme de premier cycle (baccalauréat).
                
        \item Pour qu'un cours puisse faire l'objet d'une équivalence, il doit avoir été
                    réussi avec une note égale ou supérieure à C.
                
        \item L'étudiant désireux de se voir accorder une équivalence doit remplir le
                    formulaire « Demande d'équivalence ou d'exemption de cours » (http://safire.umontreal.ca/reussite-et-ressources/ressources-etudiantes/)
                    et joindre l'original du relevé de notes officiel de l'établissement
                    universitaire. Présentez-vous au secrétariat de l'EBSI pour toute information à
                    ce sujet.
            
        \end{itemize}
    
        
    
    
        \newpage
        \section {
        Horaire de l'année 2016–2017 (1er cycle arv)
        }
        
        
        
        \textit{
        Guides > Certificat en archivistique > Horaire de l'année 2016-2017 (1er
            cycle arv) - EBSI - Université de Montréal - Confluence
        }
    
        Horaire de l'année 2016–2017 (1er cycle arv)
        
            
        \textbf{
        Automne 2016
        }
    
            Rentrée sans cours : Mercredi 31 août 2016
            Début des cours : Jeudi 1er septembre 2016
            
                
                    
                        
                             
                            Lundi
                            Mardi
                            Mercredi
                            Jeudi
                            Vendredi
                        
                        
                            13h-16h
                             
                             
                             
                             
                            
                                
                                    ARV 3053 (à option)
                                
                                
                                    Archives et information : 
                                
                                
                                    aspects juridiques
                                
                                Marie Demoulin
                                Salle : B-3255
                                Pav. 3200 Jean-Brillant
                            
                        
                        
                            16h-19h
                            
                                
                                    ARV1052 (obligatoire)
                                
                                
                                    Typologie des archives
                                
                                Aïda Chebbi
                                Salle : B-3335Jeudi 8 déc. :
                                    B-4345
                                Pav. 3200 Jean-Brillant
                            
                            
                                
                                    ARV1050 (obligatoire)
                                
                                
                                    Introduction à l'archivistique
                                
                                Daniel Ducharme
                                Salle : G-1015
                                Pavillon Roger-Gaudry
                            
                             
                            
                                
                                    ARV1054 (obligatoire)
                                
                                
                                    Classification des archives
                                
                                Sabine Mas
                                Salle : B-3345
                                Pav. 3200 Jean-Brillant
                            
                            
                                 
                            
                        
                        
                            19h-22h
                            
                                 
                                 
                            
                             
                            
                                
                                    ARV3051 (à option)
                                
                                
                                    Préservation des archives
                                
                                Dominique Plante
                                Salle : B-3215
                                Pav. 3200 Jean-Brillant
                            
                            
                                
                                    ARV1055 (obligatoire)
                                
                                
                                    Description des archives
                                
                                François Cartier
                                Salle : B-3345
                                Pav. 3200 Jean-Brillant
                            
                             
                        
                    
                
            
            
        \textbf{
        Hiver 2017 (provisoire)
        }
    
            Début des cours : Jeudi 5 janvier 2017
            
                
                    
                        
                             
                            Lundi
                            Mardi
                            Mercredi
                            Jeudi
                            Vendredi
                        
                        
                            
                                16h-19h
                            
                            
                                
                                    ARV1058 (obligatoire)
                                
                                
                                    Administration 
                                
                                
                                    des archives
                                
                                Chargé de cours
                                Salle : XXXX
                            
                            
                                
                                    ARV1050 (obligatoire)
                                
                                
                                    Introduction à 
                                    l'archivistique
                                
                                Chargé de cours
                                Salle : XXXX
                            
                            
                                 ARV3054
                                            (obligatoire)
                                
                                    Gestion des archives 
                                    numériques
                                
                                Chargé de cours
                                Salle : XXXX
                            
                            
                                
                                    ARV2955 (à option)
                                
                                
                                    Histoire du livre 
                                    et 
                                
                                
                                    de l'imprimé
                                
                                Chargé de cours
                                Salle : XXXX
                            
                            
                                                                      
                            
                        
                        
                            
                                19h-22h
                            
                             
                            
                                
                                    ARV1056 (à
                                        option)
                                
                                
                                    Diffusion, communication
                                
                                
                                    et exploitation
                                
                                Yvon Lemay
                                Salle : XXXX
                            
                             
                            
                                
                                    ARV1053
                                            (obligatoire)
                                
                                
                                    Évaluation des archives
                                
                                Chargé de cours
                                Salle : XXXX
                            
                             
                        
                    
                
            
            
        \textbf{
        Été 2017 (Horaire et date à déterminer)
        }
    
            
                ARV1061 (à option)
            
            
                Archives non-textuelles
            
            Chargé de coursSalle : XXXX
             
            
                ARV1057 – Stage et ARV3052 – Activités dirigées : cours à option offerts à chaque
                    trimestre sous certaines conditions
            
            
                
                    
                
            
            Notes : Le calendrier et l'horaire peuvent être modifiés en raison
                de circonstances imprévues. Si moins de cinq étudiants s'inscrivent à un cours, le
                cours peut être annulé. L'attribution des charges de cours est faite en vertu de
                l'article 10.08 de la Convention collective des chargés de cours de l'Université de
                Montréal.
        
    
    
        \newpage
        \section {
        Certificat en gestion de l'information numérique > Horaire de l'année
            2016–20171 (1er cycle gin)
        }
        
        
        
        \textit{
        Guides > Certificat en gestion de l'information numérique > Horaire de
            l'année 2016–2017 (1er cycle gin) - EBSI - Université de Montréal -
            Confluence
        }
    
        Certificat en gestion de l'information numérique > Horaire de l'année
            2016–20171 (1er cycle gin)
        
            
        \textbf{
        Automne 2016
        }
    
            Rentrée sans cours : Mercredi 31 août 2016
            Début des cours : Jeudi 1er septembre 2016
            
                
                    
                        
                             
                            Lundi
                            Mardi
                            Mercredi
                            Jeudi
                            Vendredi
                        
                        
                            16 h à 19 h
                             
                            
                                 
                            
                             
                            
                                 
                            
                            
                                
                                    INU3051 cours à option
                                
                                
                                    Information et sites web
                                
                                Dominic Forest
                                Salle : C-2031
                                Pavillon Lionel-Groulx
                            
                        
                        
                            19 h à 22 h
                            
                                
                                    INU1010 cours obligatoire
                                
                                
                                    Création de l'information numérique
                                
                                Arnaud d'Alayer
                                Salle : B-3325
                                Pavillon Jean-Brillant
                            
                            
                                
                                    INU1001 cours obligatoire
                                
                                
                                    Introduction à l'information numérique
                                
                                Martin Bélanger (à confirmer)
                                Salle : B-3335
                                Pavillon Jean-Brillant
                            
                            
                                
                                    INU1040 cours obligatoire
                                
                                
                                    Recherche de l'info. numérique
                                
                                Stéphane Ratté
                                Salle : B-4315
                                Pavillon Roger-Gaudry
                            
                            
                                
                                    INU1020 cours obligatoire
                                
                                
                                    Organisation de l'info. numérique
                                
                                Mélissa Beaudry
                                Salle : B-4275
                                Pavillon Jean-Brillant
                            
                             
                        
                    
                
            
            
        \textbf{
        Hiver 2017 (provisoire)
        }
    
            Début des cours : Mardi 5 janvier 2016
            
                
                    
                        
                             
                            Lundi
                            Mardi
                            Mercredi
                            Jeudi
                            Vendredi
                        
                        
                            
                                16 h à 19 h
                            
                            
                                
                                    INU3011 cours à option
                                
                                
                                    Documents structurés
                                
                                Yves Marcoux
                                Salle : XXXX
                            
                            
                                
                                    INU1050 cours obligatoire
                                
                                
                                    Diffusion d'information numérique
                                
                                Dominic Forest
                                Salle : XXXX
                            
                            
                                 
                            
                            
                                 
                            
                            
                                
                                    INU3051 cours à option
                                
                                
                                    Information et sites Web
                                
                                Dominic Forest
                                Salle : XXXX
                            
                        
                        
                            
                                19 h à 22 h
                            
                             
                             
                            
                                
                                    INU1001 cours obligatoire
                                
                                
                                    Introduction à l'information numérique
                                
                                Chargé de cours
                                Salle : XXXX
                            
                            
                                
                                    INU1030 cours obligatoire
                                
                                
                                    Préservation de l'info. numérique
                                
                                Chargé de cours
                                Salle : XXXX
                            
                             
                        
                    
                
            
            
        \textbf{
        Été 2017 (Horaire et date à déterminer)
        }
    
            
                INU3052 cours à option
            
            
                Gestion de systèmes d'info. en réseau
            
            Chargé de coursSalle : XXXX
            
                
                    
                
            
            Notes :
            
        \begin{itemize}
        
                
        \item Le calendrier et l'horaire peuvent être modifiés en raison de circonstances
                    imprévues.
                
        \item Si moins de cinq étudiants s'inscrivent à un cours, le cours peut être
                    annulé.
                
        \item L'attribution des charges de cours est faite en vertu de l'article 10.08 de la
                    Convention collective des chargés de cours de l'Université de Montréal.
            
        \end{itemize}
    
        
        
            
            1 Les horaires des cours de sigle IFT et COM peuvent être consultés à
                l'URL 
        \href{
        http://admission.umontreal.ca/cours-de-1er-cycle/
        } {
        http://admission.umontreal.ca/cours-de-1er-cycle/
        }
     en effectuant une
                recherche sur le sigle désiré.
        
    
    
        \newpage
        \section {
        Admission au programme (1er cycle)
        }
        
        
        
        \textit{
        Guides > Cheminement administratif > Admission au programme (1er cycle) - EBSI
            - Université de Montréal - Confluence
        }
    
        Admission au programme (1er cycle)
        
            Vous avez reçu une lettre ou un courriel (si une adresse électronique a été fournie
                lors de la demande d'admission) du Registrariat confirmant votre admission au
                programme choisi et vous communiquant votre UNIP temporaire (notez cet
                    UNIP temporaire pour le changer lors du premier accès au Centre
                étudiant en UNIP qui vous sera utile tout au long de vos études). Ce nouvel
                    UNIP est très important, conservez-le. Il vous permettra de consulter,
                en tout temps et en toute confidentialité et sécurité, votre dossier académique :
                les résultats obtenus aux cours, votre bulletin de notes, l'état de votre
                inscription, ainsi que l'état de votre compte relatif aux droits de scolarité.
            Si vous perdez votre UNIP, accédez à Mon portail UdeM, et cliquez sur le
                lien « Oublié » de la rubrique « UNIP / mot de passe », ou
                présentez-vous au secrétariat de l'EBSI, au Pavillon Lionel-Groulx, 3150 rue
                Jean-Brillant, bureau C-2004, avec une pièce d'identité afin que l'on vous
                attribue un nouvel UNIP temporaire.
        
    
    
        \newpage
        \section {
        Inscription (1er cycle)
        }
        
        
        
        \textit{
        Guides > Cheminement administratif > Inscription (1er cycle) - EBSI -
            Université de Montréal - Confluence
        }
    
        Inscription (1er cycle)
        
            Lorsque votre planification est faite pour l'ensemble de votre programme, nous vous
                conseillons de vous inscrire à un trimestre à la fois.
            
        \textbf{
        Nouveaux étudiants
        }
    
            
        \begin{itemize}
        
                
        \item Dès le jour de la séance d'accueil, tant pour le trimestre d'automne que pour le
                    trimestre d'hiver.
            
        \end{itemize}
    
            
        \textbf{
        Étudiants déjà inscrits au programme
        }
    
            
        \begin{itemize}
        
                
        \item Au mois d'avril pour les trimestres de printemps et d'automne;
                
        \item Au mois de novembre pour le trimestre d'hiver.
            
        \end{itemize}
    
            Le choix de cours se fait par l'intermédiaire du Centre étudiant (sur le site web de
                l'Université de Montréal) ou au secrétariat de l'École (C-2004) pour les étudiants
                en probation, les étudiants désinscrits ou les étudiants souhaitant substituer un
                cours par un autre. L'inscription au cours ARV1057 se fait suite à la rencontre
                individuelle avec la coordonnatrice de stages et celle pour le ARV3052 Activités
                dirigées par le responsable du programme.
            Pour faciliter l'accès au Centre étudiant, plusieurs ordinateurs sont mis à la
                disposition des étudiants dans les bibliothèques du campus et dans les salles de
                différents pavillons (dont le pavillon 3200 Jean-Brillant).
            
        \textbf{
        Inscription par le Centre étudiant (inscription en ligne)
        }
    
            Sur la page d'accueil du site web de l'UdeM (
        \href{
        http://www.umontreal.ca/
        } {
        http://www.umontreal.ca
        }
    ), vous trouvez
                la rubrique Étudiants de l'UdeM sur laquelle vous devez cliquer.
                Sous la rubrique Mon Centre étudiant, plusieurs options s'offrent à
                vous. Cliquez alors sur Inscription. Entrez votre code d'accès et votre UNIP / mot
                de passe pour accéder à votre dossier.
            Notez que les cours du certificat en gestion de l'information
                numérique sont contingentés à 80 étudiants par cours en raison du
                nombre de places restreint des salles d'enseignement informatique. Il est important
                de vous inscrire le plus rapidement possible afin d'avoir une place aux cours que
                vous choisissez.
            Nous vous invitons à consulter le Centre étudiant régulièrement afin de prendre
                connaissance de l'état de votre dossier dont vous avez la responsabilité. Il est à
                noter qu'un service de dépannage centralisé au numéro de téléphone 514 343-7212 est
                également disponible pour répondre aux besoins des étudiants éprouvant des
                difficultés d'accès d'ordre technique ainsi que des capsules d'aide (
        \href{
        http://www.etudes.umontreal.ca/centre-etudiant/aide/index.html
        } {
        http://www.etudes.umontreal.ca/centre-etudiant/aide/index.html
        }
    ).
            Vous ne pouvez pas vous inscrire via le Centre étudiant si :
            
        \begin{itemize}
        
                
        \item Vous voulez obtenir des équivalences ou substitutions;
                
        \item Vous êtes en probation.
            
        \end{itemize}
    
            Présentez-vous au secrétariat de l'EBSI pour toute information à ce sujet.
        
    
    
        \newpage
        \section {
        Calendrier des études
        }
        
        
        
        \textit{
        Guides > Cheminement administratif > Calendrier des études - EBSI - Université
            de Montréal - Confluence
        }
    
        Calendrier des études
        
            Les dates sont importantes pour l'inscription, les modifications et l'abandon d'un
                cours pour des raisons administratives (facturation et paiement) et pédagogiques
                (réussite ou échec). Notez que toute dette antérieure bloque l'inscription à
                    un trimestre subséquent et retient le bulletin de fin de trimestre.
            Le calendrier des études 2016-2017 de la Faculté des arts et des sciences peut être
                consulté ci-dessous ainsi qu'à l'adresse 
        \href{
        http://fas.umontreal.ca/fileadmin/Documents/FAS/fas/Documents/Calendrier/Calendrier_2016-2017.pdf
        } {
        http://fas.umontreal.ca/fileadmin/Documents/FAS/fas/Documents/Calendrier/Calendrier_2016-2017.pdf
        }
    .
            
                
                    
        \href{
        /download/attachments/124097784/Calendrier_2016-2017.pdf?version=1&modificationDate=1465313682000&api=v2
        } {
        
                        
                    
        }
    
                    
                        
                            
                             PDF
                        
                    
                
            
        
    
    
        \newpage
        \section {
        Modification des cours choisis (1er cycle)
        }
        
        
        
        \textit{
        Guides > Cheminement administratif > Modification des cours choisis (1er cycle)
            - EBSI - Université de Montréal - Confluence
        }
    
        Modification des cours choisis (1er cycle)
        
            Suite à votre inscription, il est possible par le Centre étudiant (
        \href{
        http://www.etudes.umontreal.ca/
        } {
        http://www.etudes.umontreal.ca/
        }
    ) de modifier un choix de cours jusqu'à la
                date limite prévue au 
        \href{
        /pages/viewpage.action?pageId=124097784
        } {
        calendrier
                    des études
        }
    . La date exacte est clairement indiquée pour chaque trimestre.
                L'inscription déclenche automatiquement la facturation pour vos cours.
        
    
    
        \newpage
        \section {
        Facturation et paiement (1er cycle)
        }
        
        
        
        \textit{
        Guides > Cheminement administratif > Facturation et paiement (1er cycle) - EBSI
            - Université de Montréal - Confluence
        }
    
        Facturation et paiement (1er cycle)
        
            Les étudiants doivent acquitter les frais de scolarité dès réception de la facture de
                la Direction des finances même s'ils abandonnent un ou des cours par la suite.
                Habituellement, vous recevez la facture électronique dans le mois suivant
                l'inscription à partir du Centre étudiant. Pour plus d'information, les étudiants
                ont avantage à consulter le site web de l'Université de Montréal (
        \href{
        http://www.etudes.umontreal.ca/
        } {
        http://www.etudes.umontreal.ca/
        }
    ).
        
    
    
        \newpage
        \section {
        Abandon d'un cours (1er cycle)
        }
        
        
        
        \textit{
        Guides > Cheminement administratif > Abandon d'un cours (1er cycle) - EBSI
            - Université de Montréal - Confluence
        }
    
        Abandon d'un cours (1er cycle)
        
            Vous devez vous rendre au secrétariat pour signer un formulaire d'abandon. Si vous
                abandonnez un cours après le délai prévu pour la modification du
                choix de cours et l'abandon d'un cours sans frais, mais à l'intérieur de la période
                autorisée au 
        \href{
        /pages/viewpage.action?pageId=124097784
        } {
        Calendrier des
                        études (dernier jour pour abandonner un cours avec frais)
        }
    , vous
                devez signer le formulaire « Abandon de cours » disponible au secrétariat. Ceci
                signifie que vous devez assumer les droits de scolarité, mais sans pénalité
                académique : votre moyenne du trimestre ne sera pas affectée. La mention
                    ABA apparaîtra alors sur votre bulletin; on ne comptabilisera
                pas ce cours en calculant votre moyenne. Pour éviter la probation ou l'expulsion
                définitive la note de passage d'un trimestre est C et cela
                s'applique à chaque trimestre. Il y a donc une différence entre abandonner un cours
                et avoir un échec.
            Si le cours est abandonné moins d'un mois avant la fin du trimestre, l'étudiant se
                verra attribuer la note F (échec) sur son bulletin. Il est donc de
                première importance d'avertir le technicien en gestion des dossiers étudiants de
                toute décision en ce sens, même pour les cours hors-programme.
        
    
    
        \newpage
        \section {
        Aide-mémoire du cheminement administratif (1er cycle)
        }
        
        
        
        \textit{
        Guides > Cheminement administratif > Aide-mémoire du cheminement administratif
            (1er cycle) - EBSI - Université de Montréal - Confluence
        }
    
        Aide-mémoire du cheminement administratif (1er cycle)
        
            
                
                    
                        
                             
                            
                                Par le Centre étudiant
                            
                            
                                Au secrétariat
                            
                        
                        
                            Admission au programme
                            
                                
                                    ü
                                
                            
                             
                        
                        
                            Inscription
                            
                                
                                    ü
                                
                            
                             
                        
                        
                            Modifications d'inscription
                            
                                
                                    ü
                                
                            
                             
                        
                        
                            Facturation et paiement
                            
                                
                                    ü
                                
                            
                             
                        
                        
                            Abandon d'un cours
                             
                            
                                
                                    ü
                                
                            
                        
                        
                            Bulletin de notes
                            
                                
                                    ü
                                
                            
                             
                        
                        
                            Droit de scolarité
                            
                                
                                    ü
                                
                            
                             
                        
                    
                
            
        
    
    
        \newpage
        \section {
        Sources des politiques, règlements et directives (1er cycle)
        }
        
        
        
        \textit{
        Guides > Politiques, règlements et directives > Sources (1er cycle) - EBSI -
            Université de Montréal - Confluence
        }
    
        Sources des politiques, règlements et directives (1er cycle)
        
            Les extraits sont tirés des règlements suivants :
            
        \begin{itemize}
        
                
        \item Règlement des études de premier cycle. (http://www.etudes.umontreal.ca/reglements/ReglEtud1erCyc.html)
                
        \item Annuaire général de l'Université de Montréal 2015–2016. (http://www.etudes.umontreal.ca/publications/annu_pdf/2015-2016/index.html)
                
        \item Règlement disciplinaire sur le plagiat ou la fraude concernant les étudiants.(http://secretariatgeneral.umontreal.ca/fileadmin/user_upload/secretariat/doc_officiels/reglements/enseignement/ens30_3-reglement-disciplinaire-plagiat-fraude-etudiants.pdf)
                
        \item Politique de l'Université de Montréal sur la propriété intellectuelle.(http://secretariatgeneral.umontreal.ca/fileadmin/user_upload/secretariat/doc_officiels/reglements/recherche/rech60_13-politique-universite-de-montreal-propriete-intellectuelle.pdf)
                
        \item Conservation des travaux et examens.Note de service du vice-décanat aux
                    études, 5 octobre 1994
                
        \item Politique de l'EBSI sur la conservation des travaux
                
        \item Division de la gestion de documents et des archives. Règles de gestion des
                    documents (http://www.archiv.umontreal.ca/service/regles_gestion/rdg_accueil.html)
                
        \item Surveillance d'examens et plagiat.Note de service du vice-décanat aux
                    études, 6 octobre 2006.
            
        \end{itemize}
    
            En raison d'éventuelles mises à jour, la version numérique de ces règlements prévaut
                sur ce qui figure dans le présent guide.
        
    
    
        \newpage
        \section {
        Crédit et charge de travail (1er et 2e cycle)
        }
        
        
        
        \textit{
        Guides > Politiques, règlements et directives > Crédit et charge de travail
            (1er et 2e cycle) - EBSI - Université de Montréal - Confluence
        }
    
        Crédit et charge de travail (1er et 2e cycle)
        
            Le crédit est une unité qui permet à l'Université d'attribuer une valeur numérique à
                la charge de travail exigée d'un étudiant pour atteindre les objectifs d'une
                activité d'enseignement ou de recherche. Le crédit représente 45 heures consacrées
                par l'étudiant à une activité de formation incluant, s'il y a lieu, le nombre
                d'heures de travail personnel jugé nécessaire. Un cours de trois crédits représente
                donc 135 heures de travail. Ainsi, pour chaque heure de cours, l'étudiant doit
                consacrer environ deux heures de travail, d'étude ou de lectures.
        
    
    
        \newpage
        \section {
        Équivalence de cours (1er cycle)
        }
        
        
        
        \textit{
        Guides > Politiques, règlements et directives > Équivalence de cours (1er
            cycle) - EBSI - Université de Montréal - Confluence
        }
    
        Équivalence de cours (1er cycle)
        
            Il y a équivalence de cours lorsqu'un ou des cours de même niveau déjà réussis par un
                étudiant satisfont aux exigences d'un cours inscrit à son programme. 
        
    
    
        \newpage
        \section {
        Exemption de cours (1er cycle)
        }
        
        
        
        \textit{
        Guides > Politiques, règlements et directives > Exemption de cours (1er cycle)
            - EBSI - Université de Montréal - Confluence
        }
    
        Exemption de cours (1er cycle)
        
            Il y a exemption d'un cours lorsque, compte tenu de sa formation ou de son expérience
                pertinente, un candidat est autorisé à ne pas suivre un cours inscrit à son
                programme.
        
    
    
        \newpage
        \section {
        Notation (1er cycle)
        }
        
        
        
        \textit{
        Guides > Politiques, règlements et directives > Notation (1er cycle) - EBSI -
            Université de Montréal - Confluence
        }
    
        Notation (1er cycle)
        
            Les professeurs et chargés de cours de l'École de bibliothéconomie et des sciences de
                l'information utilisent le système de notation du Règlement des études de premier
                cycle (section 11.1) (
        \href{
        http://secretariatgeneral.umontreal.ca/fileadmin/secretariat/Documents/Reglements/ens30_1-reglement-etudes-premier-cycle.pdf
        } {
        http://secretariatgeneral.umontreal.ca/fileadmin/secretariat/Documents/Reglements/ens30_1-reglement-etudes-premier-cycle.pdf
        }
    ).
                À moins que l'enseignant ne spécifie autrement dans son plan de cours, les
                enseignants de l'EBSI utilisent généralement la table de concordance suivante pour
                convertir les notes chiffrées en notes littérales lorsque nécessaire. La table étant
                précise au centième près, les notes sont arrondies au centième près avant d'être
                converties en lettre (p. ex., 76,992 % sera arrondi à 76,99 %, donc B selon
                l'échelle; 76,997 % sera arrondi à 77,00 %, donc B+ selon l'échelle).
            
                
                    
                        
                            
                                Premier cycle
                            
                            
                                Note
                            
                            
                                Valeur
                            
                            
                                Pourcentage***
                            
                        
                        
                            
                                excellent
                            
                            
                                A+
                            
                            
                                4,3
                            
                            
                                90,00–100
                            
                        
                        
                            
                                A
                            
                            
                                4,0
                            
                            
                                85,00–89,99
                            
                        
                        
                            
                                A−
                            
                            
                                3,7
                            
                            
                                80,00–84,99
                            
                        
                        
                            
                                très bon
                            
                            
                                B+
                            
                            
                                3,3
                            
                            
                                77,00–79,99
                            
                        
                        
                            
                                B
                            
                            
                                3,0
                            
                            
                                73,00–76,99
                            
                        
                        
                            
                                B−
                            
                            
                                2,7
                            
                            
                                70,00–72,99
                            
                        
                        
                            
                                bon
                            
                            
                                C+
                            
                            
                                2,3
                            
                            
                                65,00–69,99
                            
                        
                        
                            
                                C
                            
                            
                                2,0*
                            
                            
                                60,00–64,99
                            
                        
                        
                            
                                C−
                            
                            
                                1,7
                            
                            
                                57,00–59,99
                            
                        
                        
                            
                                passable
                            
                            
                                D+
                            
                            
                                1,3
                            
                            
                                54,00–56,99
                            
                        
                        
                            
                                D
                            
                            
                                 1,0**
                            
                            
                                50,00–53,99
                            
                        
                        
                            
                                faible (échec)
                            
                            
                                E
                            
                            
                                0,5
                            
                            
                                35,00–49,99
                            
                        
                        
                            
                                nul (échec)
                            
                            
                                F
                            
                            
                                0,0
                            
                            
                                0–34,99
                            
                        
                        
                            
                                échec par absence
                            
                            
                                F
                            
                            
                                0,0
                            
                            
                                ABS
                            
                        
                    
                
            
            * Moyenne de passage dans un programme
                du 1er cycle
            ** Note de passage dans un cours du 1er cycle
            *** La notation par pourcentage ne fait pas l'objet d'un règlement. La correspondance
                n'est donnée ici qu'à titre indicatif. Sur le bulletin, la note finale de chaque
                cours est exprimée au moyen d'une lettre. La moyenne de l'étudiant est calculée en
                points, selon une échelle de 0,0 à 4,3, en tenant compte du nombre de crédits de
                chaque cours. La moyenne du groupe apparaît sur le bulletin.
        
    
    
        \newpage
        \section {
        Normes de succès (1er cycle)
        }
        
        
        
        \textit{
        Guides > Politiques, règlements et directives > Normes de succès (1er cycle) -
            EBSI - Université de Montréal - Confluence
        }
    
        Normes de succès (1er cycle)
        
            
        \textbf{
        Réussite d'un cours
        }
    
            L'étudiant réussit un cours lorsqu'il obtient une moyenne de D pour l'ensemble des
                examens et des travaux du cours. Un résultat inférieur à D entraîne un échec à ce
                cours. De plus, les cours de sigle ARV et INU sont soumis à la politique
                d'évaluation avec seuil : pour réussir le cours, il faut non seulement atteindre (ou
                dépasser) la note de passage sur l'ensemble des évaluations du cours, mais également
                atteindre (ou dépasser) cette même note de passage sur le sous-ensemble (pondéré)
                des activités évaluées qui sont réalisées individuellement (et non en équipe).
                Toutefois, la politique d'évaluation avec seuil pourra ne pas être appliquée pour
                les cours où les travaux individuels comptent pour moins de 50 % de la note finale.
                L'application de la politique sera alors laissée à la discrétion de
                l'enseignant.
            
        \textbf{
        Réussite dans un programme
        }
    
            L'étudiant réussit un programme et reçoit le diplôme ou le certificat qui y est
                associé s'il réussit tous les cours du programme, s'il obtient une moyenne
                cumulative d'au moins C et s'il satisfait aux autres exigences du programme.
            
        \textbf{
        Mise en probation
        }
    
            L'étudiant dont la moyenne, cumulative, annuelle ou par segment, est égale ou
                supérieure à 1,7 et inférieure à 2,0 est mis en probation.
            
        \textbf{
        Exclusion définitive
        }
    
            L'étudiant est exclu du programme dans les cas suivants :
            
        \begin{itemize}
        
                
        \item L'étudiant qui échoue à la reprise d'un cours obligatoire ou d'un cours à
                    option.
                
        \item L'étudiant dont la moyenne est inférieure à 1,7. L'étudiant exclu en cours de
                    trimestre peut être autorisé à terminer le trimestre à titre d'étudiant
                    libre.
                
        \item L'étudiant qui ne satisfait pas à toutes les conditions de sa probation dans les
                    délais prévus.
            
        \end{itemize}
    
        
    
    
        \newpage
        \section {
        Évaluation des apprentissages (1er cycle)
        }
        
        
        
        \textit{
        Guides > Politiques, règlements et directives > Évaluation des apprentissages
            (1er cycle) - EBSI - Université de Montréal - Confluence
        }
    
        Évaluation des apprentissages (1er cycle)
        
            Au début de chaque cours, le professeur informe les étudiants de la forme et des
                modalités d'évaluation du cours. Cette information figure sur le plan de cours.
            De plus, les cours obligatoires sont soumis à la politique d'évaluation avec
                    seuil : pour réussir le cours, il faut non seulement atteindre (ou
                dépasser) la note de passage (D) sur l'ensemble des évaluations du cours, mais
                également atteindre (ou dépasser) cette même note de passage sur le sous-ensemble
                (pondéré) des activités évaluées qui sont réalisées individuellement (et non en
                équipe).
        
    
    
        \newpage
        \section {
        Révision de l'évaluation (1er cycle)
        }
        
        
        
        \textit{
        Guides > Politiques, règlements et directives > Révision de l'évaluation
            (1er cycle) - EBSI - Université de Montréal - Confluence
        }
    
        Révision de l'évaluation (1er cycle)
        
            Tout étudiant peut, dans les quinze jours ouvrables suivant l'émission du relevé de
                notes, demander la révision de cette évaluation en adressant à cette fin une requête
                écrite et motivée au Directeur de l'École, ou, le cas échéant, aux autorités
                compétentes de la faculté responsable du cours.
            Si le Directeur accueille favorablement la demande, il en informe immédiatement
                l'enseignant qui doit réviser l'évaluation dans les délais fixés par le doyen;
                celle-ci peut être maintenue, diminuée ou majorée.
        
    
    
        \newpage
        \section {
        Justification d'une absence (1er cycle)
        }
        
        
        
        \textit{
        Guides > Politiques, règlements et directives > Justification d'une
            absence (1er cycle) - EBSI - Université de Montréal - Confluence
        }
    
        Justification d'une absence (1er cycle)
        
            L'étudiant doit motiver, par écrit, toute absence à une évaluation ou à un cours
                faisant l'objet d'une évaluation continue dès qu'il est en mesure de constater qu'il
                ne pourra être présent à une évaluation et fournir les pièces justificatives. Dans
                les cas de force majeure, il doit le faire le plus rapidement possible par téléphone
                ou courriel et fournir les pièces justificatives dans les cinq jours ouvrés suivant
                l'absence.
            Les pièces justificatives doivent être dûment datées et signées. De plus, le
                certificat médical doit préciser les activités auxquelles l'état de santé interdit
                de participer, la date et la durée de l'absence; il doit également permettre
                l'identification du médecin.
        
    
    
        \newpage
        \section {
        Échecs (1er cycle)
        }
        
        
        
        \textit{
        Guides > Politiques, règlements et directives > Échecs (1er cycle) - EBSI -
            Université de Montréal - Confluence
        }
    
        Échecs (1er cycle)
        
            Une note inférieure à D ou une mention E (échec) constitue un échec.
            La mention F* (échec par absence) est attribuée à l'étudiant qui ne se présente pas à
                une évaluation à moins qu'il ne justifie son absence auprès du Directeur.
            La mention F ou la note E (échec) est attribuée à l'étudiant qui, étant présent à une
                séance d'évaluation par mode d'examens, ne remet aucune copie, s'il s'agit d'une
                épreuve écrite, ou refuse de répondre aux questions, s'il s'agit d'une épreuve
                orale.
            Si l'étudiant est absent à un examen, pour un motif jugé valable par le Directeur, ce
                dernier peut soit exiger un examen différé, soit remplacer la note dudit examen par
                la note de l'examen final; il y a toujours examen différé s'il s'agit d'un examen
                final ou du seul examen de ce cours.
        
    
    
        \newpage
        \section {
        Reprises (1er cycle)
        }
        
        
        
        \textit{
        Guides > Politiques, règlements et directives > Reprises (1er cycle) - EBSI -
            Université de Montréal - Confluence
        }
    
        Reprises (1er cycle)
        
            À moins qu'il ne soit exclu définitivement d'un programme, l'étudiant qui subit un
                échec à un cours a droit de reprise sauf pour les cours éliminatoires. De façon
                générale, l'étudiant qui échoue un cours doit le reprendre ou, avec approbation de
                l'autorité compétente, lui substituer un autre cours.
        
    
    
        \newpage
        \section {
        Scolarité (1er cycle)
        }
        
        
        
        \textit{
        Guides > Politiques, règlements et directives > Scolarité (1er cycle) - EBSI -
            Université de Montréal - Confluence
        }
    
        Scolarité (1er cycle)
        
            La scolarité est le nombre minimal de trimestres durant lesquels un étudiant doit
                être inscrit pour obtenir un diplôme ou un certificat, compte tenu du nombre de
                crédits requis pour compléter un programme.
            La scolarité maximale d'un programme de certificat est de quatre années civiles à
                compter de la première inscription.
        
    
    
        \newpage
        \section {
        Attestations de fin d'études (1er et 2e cycle)
        }
        
        
        
        \textit{
        Guides > Politiques, règlements et directives > Attestations de fin
            d'études (1er et 2e cycles) - EBSI - Université de Montréal - Confluence
        }
    
        Attestations de fin d'études (1er et 2e cycle)
        
            Les attestations de fin d'étude seront émises dans des cas particuliers seulement,
                par exemple pour des fins d'emploi lorsque l'étudiant sera dans un processus formel
                de recrutement, ou dans certaines autres circonstances exceptionnelles. La direction
                de l'école verra à évaluer les exceptions. Les autres demandes seront référées au
                registrariat.
        
    
    
        \newpage
        \section {
        Consignes concernant les règles des examens (1er cycle)
        }
        
        
        
        \textit{
        Guides > Politiques, règlements et directives > Consignes concernant les règles
            des examens (1er cycle) - EBSI - Université de Montréal - Confluence
        }
    
        Consignes concernant les règles des examens (1er cycle)
        
            Lors d'un examen, les étudiants doivent se conformer aux consignes suivantes :
            
        \begin{itemize}
        
                
        \item Dépôt à l'avant de la salle de tous les effets personnels non permis pendant
                    l'examen;
                
        \item Interdiction de toute communication verbale pendant l'examen;
                
        \item Interdiction de quitter la salle pendant la première heure;
                
        \item L'étudiant qui doit s'absenter après la première heure remet sa carte d'étudiant
                    au surveillant, l'absence ne devant pas dépasser 5 minutes;
                
        \item Un seul étudiant à la fois peut quitter la salle;
                
        \item Toute infraction relative à une fraude, un plagiat ou un copiage est signalée
                    par le surveillant au Directeur ou à l'enseignant qui suspend l'évaluation.
            
        \end{itemize}
    
        
    
    
        \newpage
        \section {
        Politique concernant la conservation et la destruction des travaux et des copies
            d'examens (1er et 2e cycles)
        }
        
        
        
        \textit{
        Guides > Politiques, règlements et directives > Politique concernant la
            conservation et la destruction des travaux et des copies d'examens (1er et 2e
            cycles) - EBSI - Université de Montréal - Confluence
        }
    
        Politique concernant la conservation et la destruction des travaux et des copies
            d'examens (1er et 2e cycles)
        
            Selon la Politique de gestion de documents et des archives de l'Université de
                Montréal, « les travaux de tous genres remis à l'Université à l'occasion de cours
                suivis ou d'études faites à l'Université font partie des archives institutionnelles
                de l'Université pour des fins exclusives d'études et de recherche ». Ces documents
                sont des exemplaires principaux selon les Règles de gestion des documents
                (01720-Travaux et examens). Les travaux et les examens doivent être gardés à l'École
                pendant un an peu importe leur support. Par la suite, un échantillonnage est
                effectué selon la règle de conservation Évaluation de l'étudiant et les autres
                documents sont détruits.
            Les copies à des fins d'évaluation étant la propriété de l'Université, l'étudiant a
                donc intérêt à en conserver une copie avant de les remettre.
        
    
    
        \newpage
        \section {
        Politique concernant la consultation des travaux et copies d'examens (1er et 2e
            cycles)
        }
        
        
        
        \textit{
        Guides > Politiques, règlements et directives > Politique concernant la
            consultation des travaux et copies d'examens (1er et 2e cycles) - EBSI - Université
            de Montréal - Confluence
        }
    
        Politique concernant la consultation des travaux et copies d'examens (1er et 2e
            cycles)
        
            Les étudiants intéressés à prendre connaissance de leurs travaux ou examens peuvent
                les consulter au bureau des professeurs concernés pendant leurs heures de
                disponibilité. Pour les cours enseignés par des chargés de cours, les dates et le
                lieu de consultation seront annoncés par courriel par le secrétariat à la fin de
                chaque session. La consultation se fait dans les locaux de l'EBSI et devant témoin.
                Par ailleurs, selon les modalités établies par l'École, il est permis de faire une
                reproduction des travaux, mais non des examens en laissant une carte d'identité à la
                personne responsable de la surveillance.
        
    
    
        \newpage
        \section {
        Politique de l'Université de Montréal sur la propriété intellectuelle (1er et 2e
            cycles)
        }
        
        
        
        \textit{
        Guides > Politiques, règlements et directives > Politique de l'Université
            de Montréal sur la propriété intellectuelle (1er et 2e cycles) - EBSI - Université de
            Montréal - Confluence
        }
    
        Politique de l'Université de Montréal sur la propriété intellectuelle (1er et 2e
            cycles)
        
            Le droit d'auteur est reconnu à l'étudiant sur tout type de travail protégé par droit
                d'auteur qu'il a élaboré dans le cadre de son programme d'études à l'EBSI, selon les
                termes et conditions prévus par la Politique sur la propriété intellectuelle en
                vigueur à l'Université de Montréal. Pour la mention de droit d'auteur sur les
                travaux d'étudiants, voir la rubrique 
        \href{
        /pages/viewpage.action?pageId=124097350
        } {
        Présentation des travaux
        }
    .
        
    
    
        \newpage
        \section {
        Règlement disciplinaire sur le plagiat ou sur la fraude concernant les étudiants (1er
            et 2e cycles)
        }
        
        
        
        \textit{
        Guides > Politiques, règlements et directives > Règlement disciplinaire sur le
            plagiat ou sur la fraude concernant les étudiants (1er et 2e cycles) - EBSI - Université
            de Montréal - Confluence
        }
    
        Règlement disciplinaire sur le plagiat ou sur la fraude concernant les étudiants (1er
            et 2e cycles)
        
            Tous les étudiants doivent prendre connaissance du document « Règlement disciplinaire
                sur le plagiat ou la fraude concernant les étudiants ». Ce règlement est strictement
                appliqué pour tous les étudiants de l'EBSI (
        \href{
        http://www.etudes.umontreal.ca/publications/annu_pdf/2011-2012/reglDisc.pdf
        } {
        http://www.etudes.umontreal.ca/publications/annu_pdf/2011-2012/reglDisc.pdf
        }
    ).
        
    
    
        \newpage
        \section {
        Autres directives (1er cycle)
        }
        
        
        
        \textit{
        Guides > Politiques, règlements et directives > Autres directives (1er cycle) -
            EBSI - Université de Montréal - Confluence
        }
    
        Autres directives (1er cycle)
        
            
        \textbf{
        Présence aux cours
        }
    
            Toute l'information présentée en classe, qu'elle fasse ou non partie des notes de
                cours écrites distribuées aux étudiants, de même que toutes les lectures
                obligatoires et le contenu des travaux pratiques, sont sujets aux examens.
            
        \textbf{
        Qualité de la langue
        }
    
            L'Université de Montréal reconnaît la qualité du français comme un critère
                d'évaluation des travaux et des examens. Les enseignants en tiennent donc compte
                dans l'évaluation des travaux et des examens, et peuvent enlever jusqu'à 10 % de la
                note globale.
            
        \textbf{
        Délais et dates de remise des travaux
        }
    
            L'enseignant fixe la date de remise de chaque travail et en avertit les étudiants,
                par écrit, dès le début du trimestre. L'étudiant doit gérer son temps de façon à
                respecter toutes ses échéances et la date de remise de ses travaux. À moins
                d'indication contraire de la part de l'enseignant, les travaux doivent être remis en
                version papier. La remise de tous les travaux doit se faire au début de chaque cours
                ou séance de travaux pratiques.
            En cas de retard dans la remise d'un travail, les pénalités suivantes s'appliquent
                :
            
        \begin{itemize}
        
                
        \item Première semaine de calendrier : 5 % de la note maximale du travail retranchés
                    par jour calendaire de retard, jusqu'à concurrence de 35 %. Le jour de la date
                    prévue de la remise du travail ne compte pas. Le samedi et le dimanche
                    ainsi que les jours fériés sont comptés.
                
        \item Au-delà de ce délai : note F (échec).
            
        \end{itemize}
    
            L'étudiant qui peut donner une raison valable pour remettre un travail en retard doit
                en demander la permission par écrit à l'enseignant, avant le cours. Ce dernier doit
                décider s'il accepte ou non le retard de l'étudiant et l'en aviser par écrit. En cas
                de conflit, le litige sera tranché par le Directeur.
            Si l'enseignant accorde un délai pour la remise d'un travail, il doit indiquer une
                date limite sur l'entente qu'il remet au technicien en gestion des dossiers
                étudiants. En aucun cas, cette date ne peut excéder trois mois après la fin d'un
                trimestre. L'étudiant qui ne respecte pas ce délai devra obligatoirement se
                réinscrire au cours et payer de nouveau les frais de scolarité exigés.
            Tout travail peut être remis au secrétariat de l'EBSI. Un membre du personnel du
                secrétariat doit y apposer ses initiales et indiquer la date de réception. Après les
                heures d'ouverture, il est possible de déposer les travaux dans la boîte prévue à
                cet effet se trouvant dans le couloir d'entrée à l'extérieur du secrétariat.
            
        \textbf{
        Travaux en équipe
        }
    
            En règle générale, les travaux en équipe sont évalués globalement et tous les membres
                de l'équipe reçoivent la même évaluation. En cas de problème survenu dans le
                fonctionnement de l'équipe, l'enseignant peut demander à chacun d'identifier la part
                de travail qui lui est propre. L'enseignant se réserve le droit de vérifier et
                d'évaluer séparément chaque membre d'une équipe. En cas de problème de
                fonctionnement dans une équipe, il faut en avertir l'enseignant le plus tôt
                possible.
        
    
    
        \newpage
        \section {
         
        }
        
        
        
        \textit{
        Guides > Code d'honneur de l'EBSI - EBSI - Université de Montréal -
            Confluence
        }
    
         
        
             
            
            
        \textbf{
        Code d'honneur de
                l'EBSI
        }
    
            Une bonne proportion des gestes de plagiat provient d'une méconnaissance des
                règlements mais aussi des mauvaises pratiques de citation des sources.
            L'EBSI a adopté un code d'honneur mettant en exergue les règles qu'elle juge
                particulièrement importantes et qu'elle veut porter à la connaissance de ses
                étudiants en matière de fraude et de plagiat ainsi que par rapport aux conditions
                d'utilisation des ressources et services informatiques mis à leur disposition.
                L'École veut ainsi informer ses étudiants afin de favoriser un environnement propice
                à l'apprentissage et demande à tout nouvel étudiant inscrit à l'un des quatre
                programmes de l'École de signer ce code d'honneur pour attester qu'il en a pris
                connaissance. Une copie du code d'honneur peut être consultée à la fin de cette
                section.
            Ce code d'honneur s'inscrit dans l'initiative de l'Université de Montréal à
                promouvoir les bonnes pratiques en matière d'intégrité intellectuelle (
        \href{
        http://www.integrite.umontreal.ca
        } {
        http://www.integrite.umontreal.ca
        }
    ).
            L'étudiant qui s'inscrit à l'Université de Montréal doit respecter le cadre
                réglementaire de cet environnement d'étude et doit respecter les conditions
                d'utilisation des ressources et services mis à sa disposition.
            Les étudiants doivent être conscients qu'ils sont assujettis à l'ensemble de la
                réglementation de l'Université de Montréal, réglementation qu'ils trouveront sur le
                site du secrétariat général de l'Université : 
        \href{
        http://secretariatgeneral.umontreal.ca/
        } {
        http://secretariatgeneral.umontreal.ca/
        }
    .
            Par ce code d'honneur, l'EBSI veut informer ses étudiants afin de favoriser un
                environnement propice à l'apprentissage et leur demande d'attester qu'ils en ont
                pris connaissance. L'EBSI s'attend à ce que ses étudiants respectent ce code
                d'honneur.
             
            
                CODE D'HONNEUR
            
            
                
                    Honnêteté intellectuelle
                
            
            Tant en milieu académique qu'en milieu professionnel, l'honnêteté intellectuelle est
                une valeur centrale. Conséquemment, l'étudiant(e) se doit de respecter le Règlement
                disciplinaire sur le plagiat ou la fraude concernant les étudiants (document numéro
                30.3) de l'Université de Montréal dont voici quelques extraits :
            
                
                    Article 1 – Infractions :
                
                1.1 Constitue une infraction le fait pour un
                    étudiant de commettre une fraude ou, intentionnellement, par insouciance ou
                    négligence, tout plagiat ou copiage ainsi que :
                a) toute tentative de commettre ces actes;
                b) toute participation à ces actes;
                c) toute incitation à commettre ces actes;
                d) tout complot avec d'autres personnes en vue de
                    commettre ces actes, même s'ils ne sont pas commis ou s'ils le sont par une
                    seule des personnes ayant participé au complot.
                1.2 Constituent notamment un plagiat, copiage ou
                    fraude :
                […]
                b) l'exécution par une autre personne d'un travail
                    ou d'une activité faisant l'objet d'une évaluation, d'un rapport de stage, d'un
                    travail dirigé, d'un mémoire ou d'une thèse;
                c) l'utilisation totale ou partielle, littérale ou
                    déguisée, d'un texte d'autrui en le faisant passer pour sien ou sans indication
                    de référence à l'occasion d'un examen, d'un travail ou d'une activité faisant
                    l'objet d'une évaluation, d'un rapport de stage, d'un travail dirigé, d'un
                    mémoire ou d'une thèse;
                d) l'obtention, par vol, manœuvre ou corruption ou
                    par tout autre moyen illicite, de questions ou de réponses d'examen ou de tout
                    autre document non autorisé;
                […]
                j) la présentation, à des fins d'évaluations
                    différentes, sans autorisation, d'un même travail, travail dirigé, mémoire ou
                    thèse, intégralement ou partiellement, dans différents cours, dans différents
                    programmes de l'Université, ou à l'Université et dans un autre établissement
                    d'enseignement. »
                Article 2 – Sanctions : Les sanctions pour une infraction au règlement
                    peuvent aller de la simple réprimande, en passant par l'obtention d'une note de
                    zéro à un travail, et même aller jusqu'à l'exclusion d'un programme. De plus, «
                    [e]n cas de récidive, des sanctions plus sévères que pour une première
                    infraction peuvent être imposées; elles doivent l'être s'il s'agit d'une
                    infraction de même nature ou plus grave.
            
             
            
                
                    Environnements informatiques de l'Université de Montréal, laboratoires
                        d'informatique documentaire de l'EBSI et de la Faculté des arts et des
                        sciences (FAS)
                
            
            L'étudiant(e) s'engage à respecter les règlements en vigueur dans les laboratoires
                d'informatique documentaire de l'EBSI (locaux C 2031 et C 2035 du pavillon
                Lionel-Groulx) et de la FAS (locaux C 3001 et C 3115 du pavillon Lionel-Groulx)
                :
            
        \begin{itemize}
        
                
        \item Afin de respecter le droit d'auteur, l'étudiant(e) s'engage à
                    ne faire aucune copie des logiciels disponibles sur les postes de travail des
                    laboratoires d'informatique de l'EBSI et à ne pas installer de logiciels
                    personnels ou importés d'Internet sur les postes de travail de ces
                    laboratoires.
                
        \item Afin de protéger l'intégrité des postes, l'étudiant(e) s'engage
                    à vérifier systématiquement la présence de virus informatiques sur tout support
                    de stockage utilisé dans les laboratoires et sur tout fichier téléchargé à
                    partir du réseau Internet et à ne pas modifier les paramètres de configuration
                    des postes de travail et des différents logiciels implantés dans les
                    laboratoires.
                
        \item Afin de se conformer à la mission académique de l'Université,
                    l'étudiant(e) reconnaît que les ressources mises à sa disposition dans les
                    laboratoires d'informatique, et en particulier l'accès au web offert dans les
                    laboratoires et les graveurs de disques compacts, ne doivent être utilisés qu'à
                    des fins académiques et s'engage à respecter les règles d'utilisation des
                    environnements informatiques de l'Université. [1]
            
        \end{itemize}
    
            
                
                    Travaux d'équipe
                
            
            Tous les membres d'une équipe doivent contribuer de façon équitable aux travaux
                d'équipe et sont conjointement responsables du contenu des travaux rendus.
            
                
                    Délais et dates de remise des travaux
                
            
            Les travaux doivent être remis aux dates précisées dans les plans de cours. Les
                sanctions indiquées sous la rubrique « Politiques, règlements et directives » du
                Guide de l'étudiant s'appliqueront en cas de retard non justifié.
            
                
                    Enregistrement des cours
                
            
            La prestation des cours est soumise au droit d'auteur. Une autorisation écrite de la
                part de l'enseignant est requise pour réaliser un enregistrement audio ou vidéo d'un
                cours, même pour un usage strictement personnel.
            Les étudiants en situation en handicap doivent présenter à l'enseignant, au début du
                cours, le formulaire de mesures d'accommodement du SESH qui leur accorde le droit
                d'enregistrer les cours.
        
        
            
            [1] 
        \href{
        http://secretariatgeneral.umontreal.ca/fileadmin/secretariat/Documents/Reglements/ges40_28-politique-securite-information.pdf
        } {
        http://secretariatgeneral.umontreal.ca/fileadmin/secretariat/Documents/Reglements/ges40_28-politique-securite-information.pdf
        }
    .
        
    
    
        \newpage
        \section {
        Présentation des travaux (1er et 2e cycles)
        }
        
        
        
        \textit{
        Guides > Présentation des travaux (1er et 2e cycles) - EBSI - Université de
            Montréal - Confluence
        }
    
        Présentation des travaux (1er et 2e cycles)
        
            Les enseignants informent les étudiants des directives particulières concernant la
                présentation des travaux. Les directives ci-dessous servent de base en l'absence de
                directives particulières.
            
        \textbf{
        Éléments
                de présentation
        }
    
            Un travail écrit doit, en général, être accompagné des éléments suivants :
            
        \begin{itemize}
        
                
        \item Une page couverture comportant le nom de l'étudiant ou des étudiants, le
                    matricule de chacun des étudiants, l'avis de copyright et un pied de page
                    comprenant divers éléments d'identification du travail tel qu'illustré dans le
                    format de mise en page suggéré ci-dessous (document PDF);
            
        \end{itemize}
    
            
                
                    
        \href{
        /download/attachments/124093925/format_menp_suggere_PC.pdf?version=1&modificationDate=1467835910000&api=v2
        } {
        
                        
                    
        }
    
                    
                        
                            
                             PDF
                        
                    
                
            
            
        \begin{itemize}
        
                
        \item une table des matières, s'il y a lieu (p. ex. dans le cas d'un long
                    rapport);
                
        \item les sources consultées.
            
        \end{itemize}
    
            Les travaux écrits doivent se conformer aux directives suivantes :
            
        \begin{itemize}
        
                
        \item Les travaux étant la propriété de l'Université, l'étudiant doit s'assurer de
                    conserver en lieu sûr le fichier correspondant au document remis au
                    professeur;
                
        \item À moins d'avis contraire, les travaux remis sous forme imprimée doivent l'être
                    sur papier de format lettre (8½ × 11 po). L'étudiant doit se conformer aux
                    directives spécifiques présentées par l'enseignant.
            
        \end{itemize}
    
            
        \textbf{
        Les
                sources consultées
        }
    
            On n'insistera jamais assez sur l'importance de présenter, dans une bibliographie à
                la fin de son travail, des références bibliographiques complètes et exactes afin de
                permettre au lecteur de retrouver facilement les documents cités. Les sources
                consultées regroupent tant les sources citées dans le travail que celles consultées
                pour la rédaction du travail, mais non citées.
            Style et citations bibliographiques
            Il est important de présenter les références bibliographiques de façon uniforme.
                L'enseignant peut exiger de respecter une méthode de citation et un style
                bibliographique particulier qui est en général le style APA–Français développé et
                maintenu à jour par les bibliothèques de l'Université de Montréal et inclus dans le
                logiciel EndNote (version de la logithèque de l'UdeM). En l'absence de directives
                spécifiques, l'étudiant choisira un style cohérent et uniforme pour citer ses
                sources et rédiger ses références bibliographiques. Référez-vous au guide « Citer
                ses sources » préparé par les Bibliothèques de l'Université de Montréal
                    (
        \href{
        http://guides.bib.umontreal.ca/disciplines/22-Citer-ses-sources
        } {
        http://guides.bib.umontreal.ca/disciplines/22-Citer-ses-sources
        }
     )
                ainsi que « Citer selon les normes de l'APA » (
        \href{
        http://guides.bib.umontreal.ca/disciplines/20-Citer-selon-les-normes-de-l-APA
        } {
        http://guides.bib.umontreal.ca/disciplines/20-Citer-selon-les-normes-de-l-APA
        }
    ).
            Forme générale d'une référence bibliographique
            Vous ne trouverez ici que quelques exemples de références bibliographiques en format
                APA-Français. Référez-vous aux guides préparés par les Bibliothèques de l'Université
                de Montréal pour obtenir un éventail plus complet d'exemples.
            Monographie
                (livre)
            {Auteur(s)}. ({année}). {Titre de la monographie} ({édition}). {Lieu
                d'édition} : {Éditeur}.
            
                Canon, C. et Boulet, A. (2015). Pourquoi il ne restera plus un arbre sur la
                        terre (4e éd.). Montréal, Québec : Fides.
            
            Chapitre de livre ou Partie de monographie
            {Auteur(s) du chapitre}. ({année}). {Titre du chapitre}. Dans {Éditeur(s)
                intellectuel(s)} (dir.), {Titre de la monographie} ({édition}, {volume}, p.
                {pages}). {Lieu d'édition} : {Éditeur}.
            
                Palmer, J. K. (2001). When to stop: How to tell when you've printed too much.
                    Dans I. Treecutter (dir.), Digital Imaging and Online Text (p. 10–12).
                    Londres, Angleterre : MacMillan.
            
            Article dans une revue (périodique)
            {Auteur(s)}. ({année}). {Titre de l'article}. {Titre de la revue},
                    {Volume}({livraison}), {pages}.
            
                Ashpay, F. et Toshiba, J.-P. (1996). De moins en moins de copies.
                        Archives, 27(3), 3–9.
            
            Page web
            {Auteur(s)}. ({année}). {Titre du document}. {Format}. Repéré à {URL}
            
                Québec. Ressources naturelles et Faune. (2010). Forêts du Québec. Repéré à
                    www.mrn.gouv.qc.ca/forets/quebec/index.jsp
            
        
    
    
        \newpage
        \section {
        
        }
        
        
        
        \textit{
        Guides > Présentation des travaux (1er et 2e cycles) - EBSI - Université de
            Montréal - Confluence
        }
    
        
        
            Les enseignants informent les étudiants des directives particulières concernant la
                présentation des travaux. Les directives ci-dessous servent de base en l'absence de
                directives particulières.
            
        \textbf{
        Éléments
                de présentation
        }
    
            Un travail écrit doit, en général, être accompagné des éléments suivants :
            
        \begin{itemize}
        
                
        \item Une page couverture comportant le nom de l'étudiant ou des étudiants, le
                    matricule de chacun des étudiants, l'avis de copyright et un pied de page
                    comprenant divers éléments d'identification du travail tel qu'illustré dans le
                    format de mise en page suggéré ci-dessous (document PDF);
            
        \end{itemize}
    
            
                
                    
        \href{
        /download/attachments/124093925/format_menp_suggere_PC.pdf?version=1&modificationDate=1467835910000&api=v2
        } {
        
                        
                    
        }
    
                    
                        
                            
                             PDF
                        
                    
                
            
            
        \begin{itemize}
        
                
        \item une table des matières, s'il y a lieu (p. ex. dans le cas d'un long
                    rapport);
                
        \item les sources consultées.
            
        \end{itemize}
    
            Les travaux écrits doivent se conformer aux directives suivantes :
            
        \begin{itemize}
        
                
        \item Les travaux étant la propriété de l'Université, l'étudiant doit s'assurer de
                    conserver en lieu sûr le fichier correspondant au document remis au
                    professeur;
                
        \item À moins d'avis contraire, les travaux remis sous forme imprimée doivent l'être
                    sur papier de format lettre (8½ × 11 po). L'étudiant doit se conformer aux
                    directives spécifiques présentées par l'enseignant.
            
        \end{itemize}
    
            
        \textbf{
        Les
                sources consultées
        }
    
            On n'insistera jamais assez sur l'importance de présenter, dans une bibliographie à
                la fin de son travail, des références bibliographiques complètes et exactes afin de
                permettre au lecteur de retrouver facilement les documents cités. Les sources
                consultées regroupent tant les sources citées dans le travail que celles consultées
                pour la rédaction du travail, mais non citées.
            Style et citations bibliographiques
            Il est important de présenter les références bibliographiques de façon uniforme.
                L'enseignant peut exiger de respecter une méthode de citation et un style
                bibliographique particulier qui est en général le style APA–Français développé et
                maintenu à jour par les bibliothèques de l'Université de Montréal et inclus dans le
                logiciel EndNote (version de la logithèque de l'UdeM). En l'absence de directives
                spécifiques, l'étudiant choisira un style cohérent et uniforme pour citer ses
                sources et rédiger ses références bibliographiques. Référez-vous au guide « Citer
                ses sources » préparé par les Bibliothèques de l'Université de Montréal
                    (
        \href{
        http://guides.bib.umontreal.ca/disciplines/22-Citer-ses-sources
        } {
        http://guides.bib.umontreal.ca/disciplines/22-Citer-ses-sources
        }
     )
                ainsi que « Citer selon les normes de l'APA » (
        \href{
        http://guides.bib.umontreal.ca/disciplines/20-Citer-selon-les-normes-de-l-APA
        } {
        http://guides.bib.umontreal.ca/disciplines/20-Citer-selon-les-normes-de-l-APA
        }
    ).
            Forme générale d'une référence bibliographique
            Vous ne trouverez ici que quelques exemples de références bibliographiques en format
                APA-Français. Référez-vous aux guides préparés par les Bibliothèques de l'Université
                de Montréal pour obtenir un éventail plus complet d'exemples.
            Monographie
                (livre)
            {Auteur(s)}. ({année}). {Titre de la monographie} ({édition}). {Lieu
                d'édition} : {Éditeur}.
            
                Canon, C. et Boulet, A. (2015). Pourquoi il ne restera plus un arbre sur la
                        terre (4e éd.). Montréal, Québec : Fides.
            
            Chapitre de livre ou Partie de monographie
            {Auteur(s) du chapitre}. ({année}). {Titre du chapitre}. Dans {Éditeur(s)
                intellectuel(s)} (dir.), {Titre de la monographie} ({édition}, {volume}, p.
                {pages}). {Lieu d'édition} : {Éditeur}.
            
                Palmer, J. K. (2001). When to stop: How to tell when you've printed too much.
                    Dans I. Treecutter (dir.), Digital Imaging and Online Text (p. 10–12).
                    Londres, Angleterre : MacMillan.
            
            Article dans une revue (périodique)
            {Auteur(s)}. ({année}). {Titre de l'article}. {Titre de la revue},
                    {Volume}({livraison}), {pages}.
            
                Ashpay, F. et Toshiba, J.-P. (1996). De moins en moins de copies.
                        Archives, 27(3), 3–9.
            
            Page web
            {Auteur(s)}. ({année}). {Titre du document}. {Format}. Repéré à {URL}
            
                Québec. Ressources naturelles et Faune. (2010). Forêts du Québec. Repéré à
                    www.mrn.gouv.qc.ca/forets/quebec/index.jsp
            
        
    
    
        \newpage
        \section {
        Questions fréquemment posées (1er cycle)
        }
        
        
        
        \textit{
        Guides > Questions fréquemment posées (1er cycle) - EBSI - Université de Montréal
            - Confluence
        }
    
        Questions fréquemment posées (1er cycle)
        
            Je n'ai pas de UNIP temporaire ou je ne me souviens pas de mon UNIP permanent,
                comment puis-je en obtenir un ?
            
                Voir les rubriques 
        \href{
        /pages/viewpage.action?pageId=124097765
        } {
        Admission au
                        programme
        }
     et 
        \href{
        /pages/viewpage.action?pageId=124097246
        } {
        Accéder
                        aux services informatiques
        }
    .
            
            Les cours que j'ai complétés dans un autre programme ou dans une autre université
                peuvent-ils être considérés dans le cadre d'un certificat ?
            
                Voir les rubriques 
        \href{
        /pages/viewpage.action?pageId=124097968
        } {
        Équivalence
                        de cours
        }
     (politiques) et 
        \href{
        https://wiki.umontreal.ca/pages/viewpage.action?pageId=124097697#equivalence
        } {
        Équivalence de cours
        }
     (planification).
            
            Un étudiant à plein temps peut-il terminer un certificat en deux trimestres ?
            
                Voir la rubrique 
        \href{
        https://wiki.umontreal.ca/pages/viewpage.action?pageId=124097697#rythme
        } {
        Rythme de vos études
        }
    .
            
            Ai-je le droit de prendre un cours qui ne fait pas partie d'un certificat ?
            
                Voir la rubrique 
        \href{
        https://wiki.umontreal.ca/pages/viewpage.action?pageId=124097697#cours-choix-hp
        } {
        Cours au choix et hors-programme
        }
    .
            
            Si j'ai réussi tous mes cours, qu'est-ce qui pourrait m'empêcher d'obtenir mon
                certificat ?
            
                Voir la rubrique 
        \href{
        https://wiki.umontreal.ca/pages/viewpage.action?pageId=124097980#réussite-programme
        } {
        Réussite dans un programme
        }
    .
            
            Existe-t-il des normes pour la présentation des travaux ?
            
                Voir la rubrique 
        \href{
        /pages/viewpage.action?pageId=124097350
        } {
        Présentation
                        des travaux
        }
    .
            
            Puis-je demander une révision de l'évaluation ?
            
                Voir la rubrique 
        \href{
        /pages/viewpage.action?pageId=124097990
        } {
        Révision de
                        l'évaluation
        }
    .
            
            Comment puis-je m'impliquer dans les comités de l'École ?
            
                Voir la rubrique 
        \href{
        /pages/viewpage.action?pageId=124095459
        } {
        Composition
                        des comités
        }
    .
            
            Dois-je reprendre un cours obligatoire échoué ?
            
                Voir la rubrique 
        \href{
        /pages/viewpage.action?pageId=124097996
        } {
        Reprises
        }
    .
            
            Je dispose de combien de temps pour terminer mon certificat ?
            
                Voir la rubrique 
        \href{
        /pages/viewpage.action?pageId=124097998
        } {
        Scolarité
        }
    .
            
            Que m'arrivera-t-il si je remets un travail en retard ?
            
                Voir la rubrique 
        \href{
        https://wiki.umontreal.ca/pages/viewpage.action?pageId=124098029#délais
        } {
        Délais et dates de remise des travaux
        }
    .
            
            Je souhaite abandonner un cours, que dois-je faire ?
            
                Voir la rubrique 
        \href{
        /pages/viewpage.action?pageId=124097850
        } {
        Abandon d'un
                        cours
        }
    .
            
        
    

        \end{document}
    